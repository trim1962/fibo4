% !TeX root = appuntifibo.tex
% !BIB TS-program = biber
% !TeX encoding = UTF-8
% !TeX spellcheck = it_IT
% !TeX program = lualatex
\chapter{Matrici}
\section{Successioni di Fibonacci }
\begin{thm}[Forma vettoriale]\index{Fibonacci!vettore}
	Se $\Fib{n}$ è la successione di Fibonacci allora avremo:
	\begin{equation}
		\begin{bmatrix}
			\Fib{n}\\\Fib{n+1}
		\end{bmatrix}=\begin{bmatrix}
			0&1\\ 1&1\\
		\end{bmatrix}^n\begin{bmatrix}
		\Fib{0}\\\Fib{1}
	\end{bmatrix}
	\end{equation}
\end{thm}
\begin{thm}[Forma matriciale]\index{Fibonacci!matrice}
Se $\Fib{n}$ è la successione di Fibonacci allora avremo:
\begin{equation}
Q^n=\begin{bmatrix}
	\Fib{n+1}&\Fib{n}\\\Fib{n}&\Fib{n-1}
\end{bmatrix}=\begin{bmatrix}
	1&1\\ 1&0\\
\end{bmatrix}^n
\end{equation}
\end{thm}\cite{Gould1981}
\begin{proof}
	Dimostriamola per induzione su  $n$.
	
	Per $n=1$
	\[Q^1=\begin{bmatrix}
		\Fib{2}&\Fib{1}\\\Fib{1}&\Fib{0}
	\end{bmatrix}=\begin{bmatrix}
		1&1\\ 1&0\\
	\end{bmatrix}^1\\
	\]
	
	Supponiamola vera per $n=k$ proviamola per $n=k+1$
	\begin{align*}
		Q^k=\begin{bmatrix}
		\Fib{k+1}&\Fib{k}\\\Fib{k}&\Fib{k-1}
	\end{bmatrix}=&\begin{bmatrix}
		1&1\\ 1&0\\
	\end{bmatrix}^k\\
\intertext{Moltiplicando}
	\begin{bmatrix}
	\Fib{k+1}&\Fib{k}\\\Fib{k}&\Fib{k-1}
\end{bmatrix}\begin{bmatrix}
1&1\\ 1&0\\
\end{bmatrix}=&\begin{bmatrix}
	1&1\\ 1&0\\
\end{bmatrix}^k\begin{bmatrix}
1&1\\ 1&0\\
\end{bmatrix}\\
	\begin{bmatrix}
	\Fib{k}+\Fib{k+1}&\Fib{k+1}\\\Fib{k+1}&\Fib{k}
\end{bmatrix}=&\begin{bmatrix}
	1&1\\ 1&0\\
\end{bmatrix}^{k+1}\\
	\begin{bmatrix}
	\Fib{k+2}&\Fib{k+1}\\\Fib{k+1}&\Fib{k}
\end{bmatrix}=&\begin{bmatrix}
	1&1\\ 1&0\\
\end{bmatrix}^{k+1}=G^{k+1}\\
	\end{align*}
Cvd.
\end{proof}
Possiamo dimostrare nuovamente l'identità Cassini~\vref{thm:FibCassini}
\begin{thm}[Identità di 
Cassini]\index{Fibonacci!identità!Cassini}
	Se $\Fib{n}$ è la successione di Fibonacci allora avremo:
\[\Fib{n+1}\Fib{n-1}-\Fib{n}^2=(-1)^n\]
\end{thm}
\begin{proof}
\begin{align*}
\begin{vmatrix}
	\Fib{n+1}&\Fib{n}\\\Fib{n}&\Fib{n-1}
\end{vmatrix}=&\Fib{n+1}\Fib{n-1}-\Fib{n}^2\\
\intertext{ma}
\abs{Q^n}=&\abs{Q}^n\\
\intertext{quindi, dato che}
\begin{vmatrix}
	1&1\\ 1&0\\
\end{vmatrix}=&-1
\end{align*}
Abbiamo la tesi.
\end{proof}
\section{Successioni generalizzate}
\begin{defn}[Successioni generalizzate]\index{Successione!generalizzata!matrice}
Se $\Gib{n}$ è la successione di Fibonacci generalizzata allora avremo:
\begin{equation}
H^n=\begin{bmatrix}
	\Gib{n+1}&\Gib{n}\\\Gib{n}&\Gib{n-1}
\end{bmatrix}
\end{equation}
\end{defn}
\begin{thm}[Forma matriciale 
generalizzata]\label{thm:Formamatricialegeneralizzata}\index{Fibonacci!generalizzata!matrice}
	Se $\Gib{n}$ è la successione di Fibonacci generalizzata allora avremo:
	\begin{align*}
		H^n=&\begin{bmatrix}
			\Gib{2}&\Gib{1}\\\Gib{1}&\Gib{0}
		\end{bmatrix}\begin{bmatrix}
			1&1\\ 1&0\\
		\end{bmatrix}^{n-1}\\
	=&\begin{bmatrix}
		h+k&k\\k&h\\
	\end{bmatrix}\begin{bmatrix}
		1&1\\ 1&0\\
	\end{bmatrix}^{n-1}\\
	\end{align*}
\end{thm}
\begin{proof}
Proviamolo per induzione su $n$
\begin{align*}
\intertext{Per $n=1$ è vero infatti:}
\begin{bmatrix}
	\Gib{2}&\Gib{1}\\\Gib{1}&\Gib{0}
\end{bmatrix}=&\begin{bmatrix}
		h+k&k\\k&h\\
	\end{bmatrix}\begin{bmatrix}
		1&1\\ 1&0\\
	\end{bmatrix}^{0}\\
=&\begin{bmatrix}
	h+k&k\\k&h\\
\end{bmatrix}\begin{bmatrix}
	1&0\\ 0&1\\
\end{bmatrix}\\
\intertext{Supponiamolo vero per $n$, proviamolo per $n+1$}
\begin{bmatrix}
	\Gib{n+1}&\Gib{n}\\\Gib{n}&\Gib{n-1}
\end{bmatrix}=&\begin{bmatrix}
	h+k&k\\k&h\\
\end{bmatrix}\begin{bmatrix}
	1&1\\ 1&0\\
\end{bmatrix}^{n-1}\\
\intertext{Moltiplico ambo i lati}
\begin{bmatrix}
	\Gib{n+1}&\Gib{n}\\\Gib{n}&\Gib{n-1}
\end{bmatrix}\begin{bmatrix}
1&1\\ 1&0\\
\end{bmatrix}=&\begin{bmatrix}
	h+k&k\\k&h\\
\end{bmatrix}\begin{bmatrix}
	1&1\\ 1&0\\
\end{bmatrix}^{n-1}\begin{bmatrix}
1&1\\ 1&0\\
\end{bmatrix}\\
\begin{bmatrix}
	\Gib{n+1}+\Gib{n}&\Gib{n+1}\\\Gib{n}+\Gib{n-1}&\Gib{n}
\end{bmatrix}=&\begin{bmatrix}
	h+k&k\\k&h\\
\end{bmatrix}\begin{bmatrix}
	1&1\\ 1&0\\
\end{bmatrix}^{n}\\
\begin{bmatrix}
	\Gib{n+2}&\Gib{n+1}\\\Gib{n+1}&\Gib{n}
\end{bmatrix}=&\begin{bmatrix}
	h+k&k\\k&h\\
\end{bmatrix}\begin{bmatrix}
	1&1\\ 1&0\\
\end{bmatrix}^{n}\\
\end{align*}
cvd.
\end{proof}
\begin{cor}[Derivazione]
	Se $\Gib{n}$ è la successione di Fibonacci generalizzata allora avremo:
\begin{equation}
	H^n=\begin{bmatrix}
		h+k&k\\k&h\\
	\end{bmatrix}Q^{n-1}
\end{equation}
\end{cor}
Anche in questo caso è possibile dimostrare di nuovo 
il~\vref{thm:identitàCassinigeneralizzata} 
\begin{thm}[Identità di Cassini 
generalizzata]\label{thm:identitàCassinigeneralizzatamatrici}
	Se $\Gib{n}$ è la successione di Fibonacci generalizzata allora 
	\begin{equation}
		\Gib{n-1}\cdot\Gib{n+1}-\Gib{n}^2=(-1)^n(k^2-hk-h^2)
	\end{equation}
\end{thm}\index{Fibonacci!identità!Cassini generalizzata}
\begin{proof}
Dal~\vref{thm:Formamatricialegeneralizzata} abbiamo:
\begin{align*}
	\det H^n=\begin{vmatrix}
		\Gib{n+1}&\Gib{n}\\\Gib{n}&\Gib{n-1}
	\end{vmatrix}=&\begin{vmatrix}
		h+k&k\\k&h\\
	\end{vmatrix}\begin{vmatrix}
		1&1\\ 1&0\\
	\end{vmatrix}^{n-1}\\
\Gib{n-1}\cdot\Gib{n+1}-\Gib{n}^2=&(h^2+hk-k^2)(-1)^{n-1}\\
=&(k^2-h^2-kh)(-1)^{n}\\
\end{align*}
cvd.
\end{proof}