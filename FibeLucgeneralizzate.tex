% !TeX root = appuntifibo.tex
% !BIB TS-program = biber
% !TeX encoding = UTF-8
% !TeX spellcheck = it_IT
% !TeX program = lualatex
\chapter{Successioni generalizzate}
\section{Definizioni}
\begin{defn}[Successioni di Fibonacci generalizzata]
	Diremo successione di Fibonacci generalizzata la successione definita
	\begin{equation}
		\begin{cases}
			\Fg{0}=0\\
			\Fg{1}=1\\
			\Fg{n}=p\Fg{n-1}+\Fg{n-2}\quad n>1\\
		\end{cases}
	\end{equation} 
\end{defn}\cite{Yalciner2013}
\begin{defn}[Successioni di Lucas generalizzata]
Diremo successione di Lucas generalizzata la successione definita
\begin{equation}
	\begin{cases}
		\Lg{0}=2\\
		\Lg{1}=p\\
		\Lg{n}=p\Lg{n-1}+\Lg{n-2}\quad n>1\\
	\end{cases}
\end{equation}
\end{defn}\cite{Yalciner2013} 
\begin{thm}[Formula di Binet generalizzata]
	Se $\alpha$ e $\beta$ sono le soluzioni dell'equazione \begin{equation}
		x^2-px-1=0
	\end{equation} allora la formula di Binet per la successione
 generalizzata 
	di Fibonacci è \begin{equation}
	\Fg{n}=\dfrac{\alpha^n-\beta^n}{\alpha-\beta}
\end{equation}
mentre \begin{equation}
	\Lg{n}=\alpha^n+\beta^n
\end{equation}
è la formula di Binet per la successione generalizzata di Lucas 
\end{thm}\index{Formula!Binet!Fibonacci 
generalizzata}\index{Formula!Binet!Lucas generalizzata}\cite{Yalciner2013} 
\begin{lem}[Proprietà]
	Se $\alpha$ e $\beta$ sono le soluzioni dell'equazione \begin{equation}
	x^2-px-1=0
\end{equation} allora:
\begin{align*}
	\alpha\beta=&-1\\
	\alpha+\beta=&p\\
	\alpha-\beta=&\sqrt{p^2+4}\\
	p-2\beta=&\sqrt{p^2+4}\\
	2\alpha-p =&\sqrt{p^2+4}\\
\end{align*}
\end{lem}