% !TeX root = appuntifibo.tex
% !BIB TS-program = biber
% !TeX encoding = UTF-8
% !TeX spellcheck = it_IT
% !TeX program = lualatex
\chapter{Successioni generalizzate}
\section{Definizioni}
\begin{defn}[Successioni di Fibonacci generalizzata]
	Diremo successione di Fibonacci generalizzata la successione definita
	\begin{equation}
		\begin{cases}
			\FG{0}=0\\
			\FG{1}=1\\
			\FG{n}=p\FG{n-1}+\FG{n-2}\quad n>1\\
		\end{cases}
	\end{equation} 
\end{defn}\cite{Yalciner2013}
\begin{defn}[Successioni di Lucas generalizzata]
Diremo successione di Lucas generalizzata la successione definita
\begin{equation}
	\begin{cases}
		\LG{0}=2\\
		\LG{1}=p\\
		\LG{n}=p\LG{n-1}+\LG{n-2}\quad n>1\\
	\end{cases}
\end{equation}
\end{defn}\cite{Yalciner2013} 
\begin{lem}[Proprietà]
	Se $\alpha$ e $\beta$ sono le soluzioni dell'equazione \begin{equation}
		x^2-px-1=0
	\end{equation} allora:
	\begin{align*}
		\alpha\beta=&{}-1\\
		\alpha+\beta=&{}p\\
		\alpha-\beta=&{}\sqrt{p^2+4}\\
		p-2\beta=&{}\sqrt{p^2+4}\\
		2\alpha-p =&{}\sqrt{p^2+4}\\
	\end{align*}
\end{lem}  
\begin{thm}[Formula di Binet generalizzata]
	Se $\alpha$ e $\beta$ sono le soluzioni dell'equazione \begin{equation}
		x^2-px-1=0
	\end{equation} allora la formula di Binet per la successione
 generalizzata 
	di Fibonacci è \begin{equation}
	\FG{n}=\dfrac{\alpha^n-\beta^n}{\alpha-\beta}
\end{equation}
mentre \begin{equation}
	\LG{n}=\alpha^n+\beta^n
\end{equation}
è la formula di Binet per la successione generalizzata di Lucas 
\end{thm}\index{Formula!Binet!Fibonacci 
generalizzata}\index{Formula!Binet!Lucas generalizzata}\cite{Yalciner2013} 
\begin{proof}
	\begin{align*}
		\intertext{Poniamo:}
		&\FG{n}={}x\alpha^n+y\beta^n\\
		&\left\{
	\begin{aligned}
			\alpha^0x+\beta^0y=0\\ \alpha^1y+\beta^1y=1
		\end{aligned}\right.
		\intertext{Otteniamo}
		&\left\{
	\begin{aligned}
			x=&\dfrac{1}{\alpha-\beta}\\ 
			{\hspace{1.5em}}\\y=&-\dfrac{1}{\alpha-\beta}
		\end{aligned}\right.\\
		%\intertext{Per il~\vref{lem:FibpropPhi}}
		%&\left\{
		%\begin{aligned}
			%	x=\dfrac{a}{\alpha-\beta}\\ y=-\dfrac{b}{\alpha-\beta}
			%\end{aligned}\right.\\ 
			&\FG{n}={}\dfrac{\alpha^n-\beta^n}{\alpha-\beta}
		\end{align*}
		quindi
		\begin{equation}
			\FG{n}=\dfrac{1}{\sqrt{P^2+4}}
			\left[\left(\dfrac{p+\sqrt{p^2+4}}{2}\right)^n-\left(\dfrac{p-\sqrt{p^2+4}}{2}\right)^n\right]
		\end{equation}
		cvd.
	\end{proof}
%%%%%%%%%%%%%%%%%%%%%%%%%%%%%%%%%%%%%%%%%%%%%%%
\section{Definizioni}
\begin{defn}[Successioni di Fibonacci generalizzata]
	Diremo successione di Fibonacci generalizzata la successione definita
	\begin{equation}
		\begin{cases}
			\Fg{0}=0\\
			\Fg{1}=1\\
			\Fg{n}=p\Fg{n-1}-q\Fg{n-2}\quad n>1\\
		\end{cases}
	\end{equation} 
\end{defn}\cite{Rabinowitz_1996}
\begin{defn}[Successioni di Lucas generalizzata]
	Diremo successione di Lucas generalizzata la successione definita
	\begin{equation}
		\begin{cases}
			\Lg{0}=2\\
			\Lg{1}=p\\
			\Lg{n}=p\Lg{n-1}-q\Lg{n-2}\quad n>1\\
		\end{cases}
	\end{equation}
\end{defn}\cite{Rabinowitz_1996}
\begin{lem}[Proprietà]
	Se $r_{1}$ e $r_{2}$ sono le soluzioni dell'equazione \begin{equation}
		x^2-px+q=0
	\end{equation} allora:
	\begin{align*}
		r_{1}r_{2}=&{}q\\
		r_{1}+r_{2}=&{}p\\
		r_{1}-r_{2}=&{}\sqrt{p^2-4q}\\
		p-2r_{2}=&{}\sqrt{p^2-4q}\\
		2r_{1}-p =&{}\sqrt{p^2-4q}\\
	\end{align*}
\end{lem}  
\begin{thm}[Formula di Binet 
generalizzata]\label{thm:FormulaBinetgeneralizzatarr}
	Se $r_{1}$ e $r_{2}$ sono le soluzioni dell'equazione \begin{equation}
		x^2-px+q=0
	\end{equation} allora la formula di Binet per la successione
	generalizzata 
	di Fibonacci è \begin{equation}
		\Fg{n}=\dfrac{r_{1}^n-r_{2}^n}{r_{1}-r_{2}}
	\end{equation}
	mentre \begin{equation}
		\Lg{n}=r_{1}^n+r_{2}^n
	\end{equation}
	è la formula di Binet per la successione generalizzata di Lucas 
\end{thm}\index{Formula!Binet!Fibonacci 
	generalizzata}\index{Formula!Binet!Lucas 
	generalizzata}\cite{Rabinowitz_1996}
\begin{proof}
		\proofpart{Prima dimostrazione}
	\begin{align*}
		\intertext{Poniamo:}
		\Fg{n}=&{}xr_{1}^n+yr_{2}^n\\
		&\left\{
	\begin{aligned}
			r_{1}^0x+r_{2}^0y=0\\ r_{1}^1y+r_{2}^1y=1
		\end{aligned}\right.
		\intertext{Otteniamo}
		&\left\{
	\begin{aligned}
			x=&\dfrac{1}{r_{1}-r_{2}}\\ 
			y=&-\dfrac{1}{r_{1}-r_{2}}
		\end{aligned}\right.\\
		%\intertext{Per il~\vref{lem:FibpropPhi}}
		%&\left\{
		%\begin{aligned}
			%	x=\dfrac{a}{\alpha-\beta}\\ y=-\dfrac{b}{\alpha-\beta}
			%\end{aligned}\right.\\ 
			\Fg{n}=&{}\dfrac{r_{1}^n-r_{2}^n}{r_{1}-r_{2}}
		\end{align*}
		quindi
		\begin{equation}
			\Fg{n}=\dfrac{1}{\sqrt{p^2-4q}}
			\left[\left(\dfrac{p+\sqrt{p^2-4q}}{2}\right)^n
			-\left(\dfrac{p-\sqrt{p^2-4q}}{2}\right)^n\right]
		\end{equation}
			\proofpart{Seconda dimostrazione}
				\begin{align*}
				\intertext{Poniamo che:}
				\Lg{n}=&{}xr_{1}^n+yr_{2}^n\\
				&\left\{
				\begin{aligned}
					r_{1}^0x+r_{2}^0y=2\\ 
					r_{1}^1y+r_{2}^1y=p
				\end{aligned}\right.
				\intertext{Otteniamo}
				&\left\{
				\begin{aligned}
					x=&\dfrac{p-2r_{2}}{r_{1}-r_{2}}=1\\ 
					y=&-\dfrac{2r_{1}-p}{r_{1}-r_{2}}=1
				\end{aligned}\right.\\
				%\intertext{Per il~\vref{lem:FibpropPhi}}
				%&\left\{
				%\begin{aligned}
				%	x=\dfrac{a}{\alpha-\beta}\\ y=-\dfrac{b}{\alpha-\beta}
				%\end{aligned}\right.\\ 
				\Lg{n}={}&r_{1}^n+r_{2}^n
			\end{align*}
		quindi
		\begin{equation}
			\Lg{n}=
		\left(\dfrac{p+\sqrt{p^2-4q}}{2}\right)^n
			+\left(\dfrac{p-\sqrt{p^2-4q}}{2}\right)^n
		\end{equation}
	\end{proof}
\begin{thm}[Algoritmo per rimuovere $r_1$ e 
$r_2$]~\cite{Rabinowitz_1996}\label{thm:FibLucRimuovirr}
	Se $\Fg{n}$ è la successione di Fibonacci e  $\Lg{n}$ è quella di Lucas 
	allora:\begin{equation}
		\left\{\begin{aligned}
			r_1^n={}&\dfrac{\Lg{n}+(\sqrt{p^2-4q})\Fg{n}}{2}\\
			r_2^n={}&\dfrac{\Lg{n}-(\sqrt{p^2-4q})\Fg{n}}{2}\\
		\end{aligned}
		\right.
	\end{equation}
\end{thm}
\begin{proof}
	Utilizzando il~\vref{thm:FormulaBinetgeneralizzatarr} 
	possiamo scrivere
	\begin{equation*}
		\left\{
		\begin{aligned}
			\Fg{n}=&\dfrac{r_{1}^n-r_{2}^n}{r_{1}-r_{2}}\\
			\Lg{n}=&r_{1}^n+r_{2}^n
		\end{aligned}
		\right.
	\end{equation*}
	che risolto rispetto $r_1^n$ e $r_2^n$ da la tesi.
\end{proof}