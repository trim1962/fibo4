% !TeX root = appuntifibo.tex
% !BIB TS-program = biber
% !TeX encoding = UTF-8
% !TeX spellcheck = it_IT
% !TeX program = lualatex
\chapter{Successioni generalizzate}
\section{Definizioni}
\begin{defn}[Successioni di Fibonacci generalizzata]
	Diremo successione di Fibonacci generalizzata la successione definita
	\begin{equation}
		\begin{cases}
			\FG{0}=0\\
			\FG{1}=1\\
			\FG{n}=P\FG{n-1}+\FG{n-2}\quad n>1\\
		\end{cases}
	\end{equation} 
\end{defn}\cite{Yalciner2013}
\begin{defn}[Successioni di Lucas generalizzata]
Diremo successione di Lucas generalizzata la successione definita
\begin{equation}
	\begin{cases}
		\LG{0}=2\\
		\LG{1}=P\\
		\LG{n}=P\LG{n-1}+\LG{n-2}\quad n>1\\
	\end{cases}
\end{equation}
\end{defn}\cite{Yalciner2013} 
\begin{lem}[Proprietà]
	Se $\alpha$ e $\beta$ sono le soluzioni dell'equazione \begin{equation}
		x^2-Px-1=0
	\end{equation} allora:
	\begin{align*}
		\alpha\beta=&{}-1\\
		\alpha+\beta=&{}P\\
		\alpha-\beta=&{}\sqrt{P^2+4}\\
		P-2\beta=&{}\sqrt{P^2+4}\\
		2\alpha-P =&{}\sqrt{P^2+4}\\
	\end{align*}
\end{lem}  
\begin{thm}[Formula di Binet generalizzata]
	Se $\alpha$ e $\beta$ sono le soluzioni dell'equazione \begin{equation}
		x^2-Px-1=0
	\end{equation} allora la formula di Binet per la successione
 generalizzata 
	di Fibonacci è \begin{equation}
	\FG{n}=\dfrac{\alpha^n-\beta^n}{\alpha-\beta}
\end{equation}
mentre \begin{equation}
	\LG{n}=\alpha^n+\beta^n
\end{equation}
è la formula di Binet per la successione generalizzata di Lucas 
\end{thm}\index{Formula!Binet!Fibonacci 
generalizzata}\index{Formula!Binet!Lucas generalizzata}\cite{Yalciner2013} 
\begin{proof}
	\begin{align*}
		\intertext{Poniamo:}
		&\FG{n}={}x\alpha^n+y\beta^n\\
		&\left\{
	\begin{aligned}
			\alpha^0x+\beta^0y=0\\ \alpha^1y+\beta^1y=1
		\end{aligned}\right.
		\intertext{Otteniamo}
		&\left\{
	\begin{aligned}
			x=&\dfrac{1}{\alpha-\beta}\\ 
			{\hspace{1.5em}}\\y=&-\dfrac{1}{\alpha-\beta}
		\end{aligned}\right.\\
		%\intertext{Per il~\vref{lem:FibpropPhi}}
		%&\left\{
		%\begin{aligned}
			%	x=\dfrac{a}{\alpha-\beta}\\ y=-\dfrac{b}{\alpha-\beta}
			%\end{aligned}\right.\\ 
			&\FG{n}={}\dfrac{\alpha^n-\beta^n}{\alpha-\beta}
		\end{align*}
		quindi
		\begin{equation}
			\FG{n}=\dfrac{1}{\sqrt{P^2+4}}
			\left[\left(\dfrac{p+\sqrt{p^2+4}}{2}\right)^n-\left(\dfrac{p-\sqrt{p^2+4}}{2}\right)^n\right]
		\end{equation}
		cvd.
	\end{proof}
%%%%%%%%%%%%%%%%%%%%%%%%%%%%%%%%%%%%%%%%%%%%%%%
\section{Definizioni}
\begin{defn}[Successioni di Fibonacci generalizzata]
	Diremo successione di Fibonacci generalizzata la successione definita
	\begin{equation}
		\begin{cases}
			\Fg{0}=0\\
			\Fg{1}=1\\
			\Fg{n}=P\Fg{n-1}-Q\Fg{n-2}\quad n>1\\
		\end{cases}
	\end{equation} 
\end{defn}\cite{Rabinowitz_1996}
\begin{defn}[Successioni di Lucas generalizzata]
	Diremo successione di Lucas generalizzata la successione definita
	\begin{equation}
		\begin{cases}
			\Lg{0}=2\\
			\Lg{1}=P\\
			\Lg{n}=P\Lg{n-1}-Q\Lg{n-2}\quad n>1\\
		\end{cases}
	\end{equation}
\end{defn}\cite{Rabinowitz_1996}
\begin{lem}[Proprietà]\label{lem:FibonacciGenLemma}
	Se $r_{1}$ e $r_{2}$ sono le soluzioni dell'equazione \begin{equation}
		x^2-Px+Q=0
	\end{equation} allora:
	\begin{align*}
		r_{1}r_{2}=&{}Q\\
		r_{1}+r_{2}=&{}P\\
		P^2-4Q=&{}D\\
		r_{1}-r_{2}=&{}\sqrt{D}\\
		P-2r_{2}=&{}\sqrt{D}\\
		2r_{1}-P=&{}\sqrt{D}\\
	\end{align*}
\end{lem}  
\begin{thm}[Formula di Binet 
generalizzata]\label{thm:FormulaBinetgeneralizzatarr}
	Se $r_{1}$ e 	$r_{2}$ sono le soluzioni dell'equazione \begin{equation}
		x^2-Px+Q=0
	\end{equation} allora la formula di Binet per la successione
	generalizzata 
	di Fibonacci è \begin{equation}
		\Fg{n}=\dfrac{r_{1}^n-r_{2}^n}{r_{1}-r_{2}}
	\end{equation}
	mentre \begin{equation}
		\Lg{n}=r_{1}^n+r_{2}^n
	\end{equation}
	è la formula di Binet per la successione generalizzata di Lucas 
\end{thm}\index{Formula!Binet!Fibonacci 
	generalizzata}\index{Formula!Binet!Lucas 
	generalizzata}\cite{Rabinowitz_1996}
\begin{proof}
		\proofpart{Prima dimostrazione}
	\begin{align*}
		\intertext{Poniamo:}
		\Fg{n}=&{}xr_{1}^n+yr_{2}^n\\
		&\left\{
	\begin{aligned}
			r_{1}^0x+r_{2}^0y=0\\ r_{1}^1y+r_{2}^1y=1
		\end{aligned}\right.
		\intertext{Otteniamo}
		&\left\{
	\begin{aligned}
			x=&\dfrac{1}{r_{1}-r_{2}}\\ 
			y=&-\dfrac{1}{r_{1}-r_{2}}
		\end{aligned}\right.\\
		%\intertext{Per il~\vref{lem:FibpropPhi}}
		%&\left\{
		%\begin{aligned}
			%	x=\dfrac{a}{\alpha-\beta}\\ y=-\dfrac{b}{\alpha-\beta}
			%\end{aligned}\right.\\ 
			\Fg{n}=&{}\dfrac{r_{1}^n-r_{2}^n}{r_{1}-r_{2}}
		\end{align*}
		quindi
		\begin{equation}
			\Fg{n}=\dfrac{1}{\sqrt{D}}
			\left[\left(\dfrac{P+\sqrt{D}}{2}\right)^n
			-\left(\dfrac{p-\sqrt{D}}{2}\right)^n\right]
		\end{equation}
			\proofpart{Seconda dimostrazione}
				\begin{align*}
				\intertext{Poniamo che:}
				\Lg{n}=&{}xr_{1}^n+yr_{2}^n\\
				&\left\{
				\begin{aligned}
					r_{1}^0x+r_{2}^0y=2\\ 
					r_{1}^1y+r_{2}^1y=P
				\end{aligned}\right.
				\intertext{Otteniamo}
				&\left\{
				\begin{aligned}
					x=&\dfrac{P-2r_{2}}{r_{1}-r_{2}}=1\\ 
					y=&-\dfrac{2r_{1}-P}{r_{1}-r_{2}}=1
				\end{aligned}\right.\\
				%\intertext{Per il~\vref{lem:FibpropPhi}}
				%&\left\{
				%\begin{aligned}
				%	x=\dfrac{a}{\alpha-\beta}\\ y=-\dfrac{b}{\alpha-\beta}
				%\end{aligned}\right.\\ 
				\Lg{n}={}&r_{1}^n+r_{2}^n
			\end{align*}
		quindi
		\begin{equation}
			\Lg{n}=
		\left(\dfrac{P+\sqrt{D}}{2}\right)^n
			+\left(\dfrac{P-\sqrt{D}}{2}\right)^n
		\end{equation}
	\end{proof}
\begin{thm}[Algoritmo per rimuovere $r_1$ e 
$r_2$]~\cite{Rabinowitz_1996}\label{thm:FibLucRimuovirr}
	Se $\Fg{n}$ è la successione di Fibonacci e  $\Lg{n}$ è quella di Lucas 
	allora:\begin{equation}
		\left\{\begin{aligned}
			r_1^n={}&\dfrac{\Lg{n}+(\sqrt{D})\Fg{n}}{2}\\
			r_2^n={}&\dfrac{\Lg{n}-(\sqrt{D})\Fg{n}}{2}\\
		\end{aligned}
		\right.
	\end{equation}
\end{thm}
\begin{proof}
	Utilizzando il~\vref{thm:FormulaBinetgeneralizzatarr} 
	possiamo scrivere
	\begin{equation*}
		\left\{
		\begin{aligned}
			\Fg{n}=&\dfrac{r_{1}^n-r_{2}^n}{r_{1}-r_{2}}\\
			\Lg{n}=&r_{1}^n+r_{2}^n
		\end{aligned}
		\right.
	\end{equation*}
	che risolto rispetto $r_1^n$ e $r_2^n$ da la tesi.
\end{proof}
\begin{thm}[Algoritmo per rimuovere potenze di $r_1$ e 
	$r_2$]~\cite{Rabinowitz_1996}\label{thm:FibLucRimuovirrpotenze}
	Se $\Fg{n}$ è la successione di Fibonacci 
	allora:
		\begin{align}
		r_1^n={}&r_1\Fg{n}-Q\Fg{n-1}\\
		r_2^n={}&r_2\Fg{n}-Q\Fg{n-1}\\
		\end{align}
\end{thm}
\begin{proof}
	Dimostriamo la prima relazione per induzione su n
	\begin{align*}
	r_1^1={}&r_1\Fg{1}-Q\Fg{0}\\
	r_1^1={}&r_1\\
	\intertext{Poniamo che sia vera per $n-1$ e proviamolo per $n$}
	r_1^{n-1}={}&r_1\Fg{n-1}-Q\Fg{n-2}\\
	r_1^{n}={}&r_1^2\Fg{n-1}-Qr_1\Fg{n-2}\\
	={}&(Pr_1-Q)\Fg{n-1}-Qr_1\Fg{n-2}\\
	={}&Pr_1\Fg{n-1}-Qr_1\Fg{n-2}-Q\Fg{n-1}\\
	={}&r_1(P\Fg{n-1}-Q\Fg{n-2})-Q\Fg{n-1}\\
	={}&r_1P\Fg{n}-Q\Fg{n-1}\\
	\end{align*}
da cui la tesi. La seconda è analoga.
\end{proof}
\begin{thm}[Successioni indici 
negativi]\label{thm:FormulaBinetgeneralizzatavalneg}~\cite{Rabinowitz_1996}
	Se $\Fg{n}$ è la successione di Fibonacci e  $\Lg{n}$ è quella di Lucas 
	allora : \begin{equation}
		\Fg{-n}=-\dfrac{1}{Q^n}	\Fg{n}
	\end{equation}
	mentre \begin{equation}
		\Lg{-n}=\dfrac{1}{Q^n}	\Lg{n}
	\end{equation}
\end{thm}
\begin{proof}
		\proofpart{Prima parte}
	Dal~\vref{thm:FormulaBinetgeneralizzatarr} abbiamo
	\begin{align*}
		\Fg{-n}=&{}\dfrac{r_{1}^{-n}-r_{2}^{-n}}{r_{1}-r_{2}}\\
		=&{}\dfrac{(r_1r_2)^{-n}(r_{1}^{n}-r_{2}^{n})}{r_{2}-r_{1}}\\
		=&{}-Q^{-n}\Fg{n}\\
		=&{}-\dfrac{\Fg{n}}{Q^{n}}  \\
	\end{align*}
\proofpart{Seconda parte}
	Dal~\vref{thm:FormulaBinetgeneralizzatarr} abbiamo
\begin{align*}
	\Lg{-n}=&{}r_{1}^{-n}+r_{2}^{-n}\\
	=&{}(r_1r_2)^{-n}(r_{1}^{n}+r_{2}^{n})\\
	=&{}(Q)^{-n}\Lg{n}\\
	=&{}\dfrac{\Lg{n}}{Q^n}
\end{align*}
\end{proof}
\section{Formule di addizione e sottrazione}
\begin{thm}[Formule di 
addizione]~\cite{Rabinowitz_1996}\label{thm:LucFibSommaG}
	Se $\Fg{n}$ è la successione di Fibonacci e  $\Lg{n}$ è quella di Lucas 
	allora:
	\begin{align}
		\Fg{n+m}={}&\dfrac{\Fg{m}\Lg{n}+\Lg{m}\Fg{n}}{2}\label{eqn:FibLucSommaprodotto}\\
		\Lg{n+m}={}&\dfrac{5\Fg{m}\Fg{n}+\Lg{m}\Lg{n}}{2}\label{eqn:LucFibSommaprodotto}
	\end{align}
\end{thm}
\begin{proof}
	\proofpart{Fibonacci}
	\begin{align*}
		\Fg{m}\Lg{n}+\Lg{m}\Fg{n}={}&\dfrac{1}{a-b}[r_1^m-r_2^m][r_1^n+r_2^n]+[r_1^m+r_2^m]\dfrac{1}{r_1-r_2}[r_1^n-r_2^n]\\
		={}&\dfrac{2}{r_1-r_2}[r_1^{n+m}-r_2^{n+m}]\\
		={}&2\Fg{m+n}\\
	\end{align*}
	\proofpart{Lucas}
	\begin{align*}
		D\Fg{m}\Fg{n}+\Lg{m}\Lg{n}=
		&\dfrac{r_1^{n+m}[(r_1-r_2)^2+D]}{(r_1-r_2)^2}+\dfrac{r_1^{m}r_2^{n}[(r_1-r_2)^2-D]}{(r_1-r_2)^2}\\
		+&\dfrac{r_1^{n}r_2^{m}[(r_1-r_2)^2-D]}{(r_1-r_2)^2}+\dfrac{r_2^{n+m}[(r_1-r_2)^2+D]}{(r_1-r_2)^2}
		\intertext{Per il~\vref{lem:FibonacciGenLemma}}
		={}&\dfrac{r_1^{n+m}[D+D]}{D}+\dfrac{r_2^{n+m}[D+D]}{D}\\
		={}&2(r_1^{n+m}+r_2^{n+m})\\
		={}&2\Lg{n+m}\\ 
	\end{align*}
	cvd.
\end{proof}