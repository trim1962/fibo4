% !TeX root = appuntifibo.tex
% !BIB TS-program = biber
% !TeX encoding = UTF-8
% !TeX spellcheck = it_IT
% !TeX program = lualatex
\documentclass[a4paper,oneside]{book}%
\usepackage{cmap}
\usepackage{base}
\usepackage{linebreaker}
\usepackage[big]{layaureo}
\usepackage{copyright}
\frenchspacing%
%\usepackage{amsmath}
%\usepackage{amssymb}
%\usepackage[italian]{babel}
\usepackage[thmmarks,hyperref]{ntheorem}
\usepackage{miamatematica}
\usepackage{tabelle}
%\usepackage{lmodern} % load vector font
%\usepackage[T1]{fontenc} % font encoding
%\usepackage[utf8]{inputenc} % input encoding
%\usepackage{noto}
%\usepackage[babel=true]{microtype}
\usepackage{microtype}
%\usepackage{geometry}
%\usepackage{textcomp}

%\geometry{top=1.5cm,bottom=1.5cm}
\usepackage{grafica}

%Teorema
\theoremstyle{marginbreak}
\theoremheaderfont{\normalfont\bfseries}\theorembodyfont{\slshape}
\theoremsymbol{\ensuremath{\diamondsuit}}
\theoremseparator{:} %
\newtheorem{thm}{Teorema}[section]
%Proprietà
\theoremstyle{marginbreak}
\theoremheaderfont{\normalfont\bfseries}\theorembodyfont{\slshape}
\theoremsymbol{\ensuremath{\diamondsuit}}
\theoremseparator{:}
\newtheorem{prop}{Proprietà}[section]
%lemma
\theoremstyle{changebreak}
\theoremsymbol{\ensuremath{\heartsuit}}
\theoremindent0.5cm
\theoremnumbering{greek}
\newtheorem{lem}[thm]{Lemma}
%corollario
\theoremindent0cm
\theoremsymbol{\ensuremath{\spadesuit}}
\theoremnumbering{arabic}
\newtheorem{cor}[thm]{Corollario}
%esempio
\theoremstyle{change}
\theorembodyfont{\upshape}
\theoremsymbol{\ensuremath{\ast}}
\theoremseparator{}
\newtheorem{exmp}{Esempio}[section]
%controesempio
\theoremstyle{change}
\theorembodyfont{\upshape}
\theoremsymbol{\ensuremath{\odot}}
\theoremseparator{}
\newtheorem{cexmp}{Contro esempio}[section]
%definizione
\theoremstyle{plain}
\theoremsymbol{\ensuremath{\clubsuit}}
\theoremseparator{.}
\theoremstyle{marginbreak}
%\theoremprework{\hrule\bigskip}
%\theorempostwork{\hrule\bigskip}
\newtheorem{defn}{Definizione}[section]
%commento
\theoremstyle{plain}
\theorembodyfont{\upshape}
\theoremsymbol{\ensuremath{\blacklozenge}}
\theoremseparator{:}
\newtheorem{commento}{Commento}[section]
%dimostrazione
\theoremstyle{plain}
\theoremheaderfont{\sc}
\theorembodyfont{\bfseries}
\theoremstyle{nonumberplain}
%^{}\theoremseparator{.}

\theoremsymbol{\ensuremath{\blacksquare}}
\theoremheaderfont{\bfseries}
%\theoremstyle{nonumberplain}
%\theoremstyle{marginbreak}
\theorembodyfont{\normalfont}
\newtheorem{proof}{Dimostrazione}
%\input{../Mod_base/tabelle}
\newtheorem{prob}{Problema}[section]
%Proprietà
\theoremstyle{marginbreak}
\theoremheaderfont{\normalfont\bfseries}\theorembodyfont{\slshape}
\theoremsymbol{\ensuremath{\diamondsuit}}
\theoremseparator{:}

\usepackage{pagina}

%\setlength{\headheight}{13pt}
\setlength{\headheight}{15.35403pt}
\usepackage{indice}
\usepackage{date}
\usepackage{unita_misura}


\usepackage{imakeidx}
\makeindex[options=-s ../Mod_base/oldclaudio.sti]

\usepackage{diagbox}
%\include{simboli_operatori}

\usepackage{stand_class}

\newcommand{\HRule}{\rule{\linewidth}{0.5mm}}


 \makeatletter
 \renewcommand\frontmatter{%
 	\cleardoublepage
 	\@mainmatterfalse
 	%\pagenumbering{roman}
 }
 \renewcommand\mainmatter{%
 	\cleardoublepage
 	\@mainmattertrue
 	%\pagenumbering{arabic}
 }
 \makeatother

\usepackage[grumpy,mark,markifdirty,raisemark=0.95\paperheight]{gitinfo2}
% 10/02/2018 :: 20:05:58 :: \usepackage{parskip}
\usepackage[toc,page]{appendix}

\renewcommand{\appendixtocname}{Appendici}

\renewcommand{\appendixpagename}{Appendici}


\usepackage[style=italian]{csquotes}
\usepackage[%
style=philosophy-modern,
annotation=true,
hyperref,
backend=biber,
backref]{biblatex}
\addbibresource{fibonacci.bib}
\usepackage[italian]{varioref}
\usepackage{hyperxmp}
\usepackage[pdfpagelabels]{hyperref}
\usepackage[italian,noabbrev]{cleveref}
\crefname{defn}{definizione}{definizioni}
\Crefname{defn}{Definizione}{Definizioni}
\crefname{thm}{teorema}{teoremi}
\Crefname{thm}{Teorema}{Teoremi}
\crefname{cor}{corollario}{corollari}
\Crefname{cor}{Corollario}{Corollari}
\crefname{equation}{equazione}{equazioni}
\Crefname{equation}{Equazione}{Equazioni}
\creflabelformat{equation}{#2\textup{#1}#3}
\crefname{sistema}{sistema}{sistemi}
\Crefname{sistema}{Sistema}{Sistemi}
\crefname{lem}{lemma}{lemmi}
\Crefname{lem}{Lemma}{Lemmi}
\crefname{prob}{problema}{problemi}
\Crefname{prob}{Problema}{Problemi}

\newcommand{\citaoeis}[1]{La successione è la  
\citetitle{#1}~\citeurl{#1} del OEIS}
%\usepackage{tcolorboxgest}
\title{Zibaldone di pensieri}
\author{Claudio Duchi}
\date{\datetime}
\hypersetup{%
pdfencoding=auto,
urlcolor={blue},
pdftitle={Fibonacci},
pdfsubject={Per non dimenticare},
pdfstartview={FitH},
pdfpagemode={UseOutlines},
pdflicenseurl={http://creativecommons.org/licenses/by-nc-nd/3.0/},
pdflang={it},
pdfmetalang={it},
pdfkeywords={Algebra, geometria, analisi},
pdfcopyright={Copyright (C) 2022, Claudio Duchi},
pdfcontacturl={http://breviariomatematico.altervista.org},
pdfcontactpostcode={06128},
pdfcontactphone={},
pdfcontactemail={claduc},
pdfcontactcountry={Italy},
pdfcontactcity={Perugia},
pdfcontactaddress={},
pdfcaptionwriter={Claudio Duchi},
pdfauthortitle={},%
pdfauthor={Claudio Duchi},
linkcolor={blue},
colorlinks=true,
citecolor={red},
breaklinks,
bookmarksopen,
verbose,
baseurl={http://breviariomatematico.altervista.org}
}

% !TeX root = appuntifibo.tex
% !BIB TS-program = biber
% !TeX encoding = UTF-8
% !TeX spellcheck = it_IT
% !TeX program = lualatex
\includeonly{%
fibonacci,
fibonaccigeneralizzato,
fibonaccirettangoli,
matriciFib,
FibeLucgeneralizzate,
Elenco_fibonacci,
}


%patch allieamento lista teoremi
\usepackage{regexpatch}
\makeatletter
%\xpatchcmd*{\thm@@thmline}{2.3em}{5em}{}{} % not really needed
\xpatchcmd*{\thm@@thmline@name}{2.3em}{5em}{}{} 
\xpatchcmd*{\thm@@thmline@noname}{2.3em}{5em}{}{}
\makeatother
%fine patch allieamento lista teoremi
\makeatletter
\newcounter{proofpart}
\xpretocmd{\proof}{\setcounter{proofpart}{0}}{}{}
\newcommand{\proofpart}[1]{%
	\par
	\addvspace{\medskipamount}%
	\stepcounter{proofpart}%
	\noindent\emph{Parte  \theproofpart: #1}\par\nobreak\smallskip
	\@afterheading
}
\makeatother
\usepackage{CDloghi}
\listfiles
\begin{document}
\begin{titlepage}
\begin{center}	
	\Lgrandedue\\[1cm]    
	\textsc{\LARGE Claudio Duchi}\\[1.4cm]
	\HRule \\[0.4cm]
{ \huge \bfseries FIBONACCI}\\[0.4cm]
{ \large \bfseries fritto}\\[0.4cm]
{ \huge \bfseries misto}\\[0.4cm]
\HRule \\
\vfill
	% Bottom of the page
	\polylogo[5.5]{13}		
		{\large $-$\DTMnow$-$}	
\end{center}
{\centering
Release:\gitReln\ (\gitAbbrevHash)\ Autore:\gitAuthorName\ 
\gitCommitterDate \\
}
\end{titlepage}	
	\hypersetup{pageanchor=true}
		\CDcopyright
		\tableofcontents
		\chapter*{Lista dei teoremi}
		\theoremlisttype{allname}
		\listtheorems{thm,defn,cor,comm,lem,prob}
	\addcontentsline{toc}{chapter}{\listfigurename}%
		\listoffigures
%	\addcontentsline{toc}{chapter}{\listtablename}%
%			\listoftables
			\mainmatter
% !TeX root = appuntifibo.tex
% !BIB TS-program = biber
% !TeX encoding = UTF-8
% !TeX spellcheck = it_IT
% !TeX program = lualatex
% !TeX root = appuntifibo.tex
% !BIB TS-program = biber
% !TeX encoding = UTF-8
% !TeX spellcheck = it_IT
% !TeX program = lualatex
\chapter{Numeri di Fibonacci e Lucas}
\section{Definizione}
\begin{defn}[Numeri di Fibonacci]\index{Fibonacci!definizione}
	\begin{align}
		\Fib{0}={} & 0\notag                                         \\
		\Fib{1}={} & 1\notag                                         \\
		\Fib{n}={} & \Fib{n-1}+\Fib{n-2}\quad n>2\label{eqn:Fibodef}
	\end{align}
\end{defn}\citaoeis{A000045}
\begin{defn}[Numeri di Lucas]\index{Lucas!definizione}
	\begin{align}
		\Luc{0}={} & 2\notag                                         \\
		\Luc{1}={} & 1\notag                                         \\
		\Luc{n}={} & \Luc{n-1}+\Luc{n-2}\quad n>2\label{eqn:Lucadef}
	\end{align}
\end{defn}\citaoeis{A000032}
\section{Sezione aurea}
\begin{defn}[Sezione aurea]
	Date due quantità $a$ e $b$ con $a>b>0$ diremo sezione aurea il rapporto
	\begin{equation}
		a+b:a=a:b=\varphi
	\end{equation}\label{eqn:FibAureaDef}\index{Sezione!aurea}
\end{defn}
\begin{prop}
	Dall'~\vref{eqn:FibAureaDef} abbiamo
	\begin{align}
		\dfrac{a}{b}={} & \dfrac{a+b}{a}\notag                       \\
		={}             & 1+\dfrac{b}{a}\notag                       \\
		={}             & 1+\dfrac{1}{\frac{a}{b}}\notag             \\
		\intertext{Quindi}
		\varphi={}      & 1+\dfrac{1}{\varphi}\label{eqn:FibPhiProp} \\
		\varphi^2={}    & \varphi+1\notag
	\end{align}
\end{prop}
Dall'~\vref{eqn:FibPhiProp} segue
\begin{prop}
	\begin{align}
		\varphi^2={}           & \varphi+1\notag                            \\
		\varphi^2-\varphi-1={} & 0\notag                                    \\
		x^2-x-1={}             & 0\label{eqn:FibValPhiEqua}                 \\
		x_1={}                 & \dfrac{1+\sqrt{5}}{2}\label{eqn:FibValPhi} \\
		x_1={}                 & \dfrac{1-\sqrt{5}}{2}\notag
	\end{align}
\end{prop}
L'~\vref{eqn:FibValPhi} permette di scrivere la seguente definizione
\begin{defn}[Sezione aurea]
	Diremo sezione aurea il numero
	\begin{align*}
		\varphi={}    & \dfrac{1+\sqrt{5}}{2} \\
		\intertext{inoltre vale}
		1-	\varphi={} & \dfrac{1-\sqrt{5}}{2} \\
	\end{align*}
\end{defn}\index{Sezione!aurea!valore}
\begin{lem}[Proprietà]\label{lem:FibpropPhi}
	Se $a=\varphi$ e  $b=1-\varphi$ allora
	\begin{align*}
		ab={}      & -1       \\
		a+b={}     & 1        \\
		a-b={}     & \sqrt{5} \\
		a^2+1={}   & (a-b)a   \\
		b^2+1={}   & (b-a)b   \\
		a^2={}     & a+1      \\
		b^2={}     & b+1      \\
		a^3={}     & a^2+a    \\
		b^3={}     & b^2+b    \\
		a^2+b^2={} & a+b+2    \\
		a^2b+b={}  & -a+b     \\
		-b^2a-a={} & b-a      \\
	\end{align*}
\end{lem}
\begin{proof}
	\begin{align*}
		ab={} & \dfrac{1+\sqrt{5}}{2}\dfrac{1-\sqrt{5}}{2} \\
		={}   & \dfrac{1-5}{4}                             \\
		={}   & -1
	\end{align*}
	\begin{align*}
		(a-b)a={} & a^2-ab=a^2+1 \\
		(b-a)b={} & b^2-ab=b^2+1 \\
	\end{align*}
	cvd.
\end{proof}
\section{Formula di Binet}
\begin{thm}[Formula di
		Binet]~\cite{Conti2020}\label{thm:FibFormulaBinet}\index{Formula!Binet!Fibonacci}
	Se $\Fib{n}$ è la successione di Fibonacci allora avremo:
	\begin{equation}
		\Fib{n}=\dfrac{1}{\sqrt{5}}\left[\left(\dfrac{1+\sqrt{5}}{2}\right)^n-\left(\dfrac{1-\sqrt{5}}{2}\right)^n\right]
	\end{equation}\label{eqn:FinBinet}
\end{thm}
\begin{proof}
	Poniamo $a=\varphi$ e con $b=1-\varphi$ allora
	\begin{align*}
		\intertext{Poniamo}
		a^1={}      & a\Fib{1}+\Fib{0}                 \\
		\intertext{Per induzione su $n$}
		a^{n-1}={}  & a\Fib{n-1}+\Fib{n-2}             \\
		a^{n-1}a={} & a^2\Fib{n-1}+a\Fib{n-2}          \\
		a^{n}={}    & (a+1)\Fib{n-1}+a\Fib{n-2}        \\
		a^{n}={}    & a\Fib{n-1}+\Fib{n-1}+a\Fib{n-2}  \\
		a^{n}={}    & a(\Fib{n-1}+\Fib{n-2})+\Fib{n-1} \\
		a^{n}={}    & a\Fib{n}+\Fib{n-1}               \\
		\intertext{Analogamente si dimostra}
		b^{n}={}    & b\Fib{n}+\Fib{n-1}               \\
		\intertext{Sottraendo}
		a^n-b^n={}  & a\Fib{n}-b\Fib{n}                \\
		a^n-b^n={}  & \Fib{n}(a-b)                     \\
		\Fib{n}={}  & \dfrac{a^n-b^n}{a-b}
	\end{align*}
	quindi
	\begin{equation}
		\Fib{n}=\dfrac{1}{\sqrt{5}}\left[\left(\dfrac{1+\sqrt{5}}{2}\right)^n-\left(\dfrac{1-\sqrt{5}}{2}\right)^n\right]
	\end{equation}
	cvd.
\end{proof}
Altra dimostrazione
\begin{proof}
	\begin{align*}
		\intertext{Poniamo:}
		 & \Fib{n}={}xa^n+yb^n                       \\
		 & \left\{
		\begin{aligned}
			a^0x+b^0y=0 \\ a^1y+b^1y=1
		\end{aligned}\right.
		\intertext{Otteniamo}
		 & \left\{
		\begin{aligned}
			x={} & \dfrac{1}{a-b} \\ y={}&-\dfrac{1}{a-b}
		\end{aligned}\right. \\
		%\intertext{Per il~\vref{lem:FibpropPhi}}
		%&\left\{
		%	\begin{aligned}
		%	x=\dfrac{a}{a-b}\\ y=-\dfrac{b}{a-b}
		%\end{aligned}\right.\\ 
		 & \Fib{n}={}\dfrac{a^n-b^n}{a-b}
	\end{align*}
	quindi
	\begin{equation}
		\Fib{n}=\dfrac{1}{\sqrt{5}}\left[\left(\dfrac{1+\sqrt{5}}{2}\right)^n-\left(\dfrac{1-\sqrt{5}}{2}\right)^n\right]
	\end{equation}
	cvd.
\end{proof}
\begin{thm}[Formula di Binet per
		Lucas]\label{thm:LucFormulaBinet}\index{Formula!Binet!Lucas}
	Se $\Luc{n}$ è la successione di Lucas allora avremo:
	\begin{equation}
		\Luc{n}=\left(\dfrac{1+\sqrt{5}}{2}\right)^{n}+\left(\dfrac{1-\sqrt{5}}{2}\right)^{n}
	\end{equation}\label{eqn:LucBinet}
\end{thm}
\begin{proof}
	\begin{align*}
		\intertext{Poniamo:}
		 & \Luc{n}={}xa^n+yb^n   \\
		 & \left\{
		\begin{aligned}
			a^0x+b^0y=2 \\ a^1y+b^1y=1
		\end{aligned}\right.
		\intertext{Otteniamo}
		 & \left\{
		\begin{aligned}
			x=\dfrac{1-2b}{a-b} \\ y=\dfrac{2a-1}{a-b}
		\end{aligned}\right.
		\intertext{Per il~\vref{lem:FibpropPhi}}
		 & \left\{
		\begin{aligned}
			x=1 \\ y=1
		\end{aligned}\right.     \\
		 & \Luc{n}={}a^{n}+b^{n}
	\end{align*}
	quindi
	\begin{equation}
		\Luc{n}=\left(\dfrac{1+\sqrt{5}}{2}\right)^n+\left(\dfrac{1-\sqrt{5}}{2}\right)^n
	\end{equation}
	cvd.
\end{proof}
\section{Formule di addizione}
\begin{thm}[Formule di addizione]~\cite{Rabinowitz_1996}\label{thm:LucFibSommaprodotto}
	Se $\Fib{n}$ è la successione di Fibonacci e  $\Luc{n}$ è quella di Lucas
	allora:
	\begin{align}
		\Fib{n+m}={} & \dfrac{\Fib{m}\Luc{n}+\Luc{m}\Fib{n}}{2}\label{eqn:FibLucSommaprodotto}  \\
		\Luc{n+m}={} & \dfrac{5\Fib{m}\Fib{n}+\Luc{m}\Luc{n}}{2}\label{eqn:LucFibSommaprodotto}
	\end{align}
\end{thm}
\begin{proof}
	\proofpart{Fibonacci}
	\begin{align*}
		\Fib{m}\Luc{n}+\Luc{m}\Fib{n}={} & \dfrac{1}{a-b}[a^m-b^m][a^n+b^n]+[a^m+b^m]\dfrac{1}{a-b}[a^n-b^n] \\
		={}                              & \dfrac{2}{a-b}[a^{n+m}-b^{n+m}]                                   \\
		={}                              & 2\Fib{m+n}                                                        \\
	\end{align*}
	\proofpart{Lucas}
	\begin{align*}
		5\Fib{m}\Fib{n}+\Luc{m}\Luc{n}=
		    & \dfrac{a^{n+m}[(a-b)^2+5]}{(a-b)^2}+\dfrac{a^{m}b^{n}[(a-b)^2-5]}{(a-b)^2}\\%
		+   & \dfrac{a^{n}b^{m}[(a-b)^2-5]}{(a-b)^2}+\dfrac{b^{n+m}[(a-b)^2+5]}{(a-b)^2}
		\intertext{Per il~\vref{lem:FibpropPhi}}
		={} & \dfrac{a^{n+m}[5+5]}{5}+\dfrac{b^{n+m}[5+5]}{5}                            \\
		={} & 2(a^{n+m}+b^{n+m})                                                         \\
		={} & 2\Luc{n+m}                                                                 \\
	\end{align*}
	cvd.
\end{proof}
Dal~\vref{thm:LucFibSommaprodotto} possiamo ottenere il seguente risultato
\begin{cor}[Formule di
		traslazione]~\cite{Rabinowitz_1996}\label{cor:Formuleditraslazione}
	Se $\Fib{n}$ è la successione di Fibonacci e  $\Luc{n}$ è quella di Lucas
	allora:
	\begin{align*}
		\Fib{n}={} & \dfrac{2}{5\Fib{a}\Fib{b}-\Luc{a}\Luc{b}}
		\left[\Fib{a}\Luc{n+b}-\Luc{b}\Fib{n+a}\right]         \\
		\Luc{n}={} & \dfrac{2}{5\Fib{a}\Fib{b}-\Luc{a}\Luc{b}}
		\left[5\Fib{n+a}\Fib{b}-\Luc{n+b}\Luc{a}\right]        \\
	\end{align*}
\end{cor}
\begin{proof}
	\begin{align*}
		\intertext{Utilizzando il~\vref{thm:LucFibSommaprodotto}}
		\Fib{n+m}={}                      & \dfrac{\Fib{m}\Luc{n}+\Luc{m}\Fib{n}}{2}                                                \\
		\Luc{n+m}={}                      & \dfrac{5\Fib{m}\Fib{n}+\Luc{m}\Luc{n}}{2}                                               \\
		\intertext{Ottengo}
		\Fib{n+a}={}                      & \dfrac{\Fib{a}\Luc{n}+\Luc{a}\Fib{n}}{2}                                                \\
		\Luc{n+b}={}                      & \dfrac{5\Fib{b}\Fib{n}+\Luc{b}\Luc{n}}{2}                                               \\
		\intertext{Da cui}
		\Fib{a}\Luc{n}+\Luc{a}\Fib{n}={}  & 2\Fib{n+a}                                                                              \\
		\Luc{b}\Luc{n}+5\Fib{b}\Fib{n}={} & 2\Luc{n+b}                                                                              \\
		\intertext{Risolvendo il sistema rispetto $\Fib{n}$ e $\Luc{n}$ ottengo}
		D ={}                             & \begin{vmatrix}
			                                    \Fib{a} & \Luc{a}  \\
			                                    \Luc{b} & 5\Fib{b}
		                                    \end{vmatrix}                                                                      \\ ={}& 5\Fib{a}\Fib{b}-\Luc{a}\Luc{b}\\
		D_{\Luc{n}} ={}                   & \begin{vmatrix}
			                                    2\Fib{n+a} & \Luc{a}  \\
			                                    2\Luc{n+b} & 5\Fib{b}
		                                    \end{vmatrix}                                                                   \\ ={}& 2\Fib{n+a}\Fib{b}-2\Luc{n+b}\Luc{a}\\
		D_{\Fib{n}} ={}                   & \begin{vmatrix}
			                                    \Fib{a} & 2\Fib{n+a} \\
			                                    \Luc{b} & 2\Luc{n+b}
		                                    \end{vmatrix}                                                                    \\ ={}& 2\Fib{a}\Luc{n+b}-2\Luc{b}\Fib{n+a}\\
		\Fib{n}={}                        & \dfrac{2}{5\Fib{a}\Fib{b}-\Luc{a}\Luc{b}}\left(
		\Fib{a}\Luc{n+b}-\Luc{b}\Fib{n+a}\right)                                                                                    \\
		\Luc{n}={}                        & \dfrac{2}{5\Fib{a}\Fib{b}-\Luc{a}\Luc{b}}\left(\Fib{n+a}\Fib{b}-\Luc{n+b}\Luc{a}\right) \\
	\end{align*}
	cvd.
\end{proof}
Una conseguenza del~\vref{thm:LucFibSommaprodotto} è il seguente
\begin{cor}[Formule moltiplicazione scalare]~\cite{Rabinowitz_1996}\label{cor:LucFibmoltscalare}
	Se $\Fib{n}$ è la successione di Fibonacci e  $\Luc{n}$ è quella di Lucas
	allora:
	\begin{align*}
		\Fib{kn}={} & \dfrac{\Fib{(k-1)n}\Luc{n}+
		\Luc{(k-1)n}\Fib{n}}{2}\label{eqn:Fibmoltiplicazionescalare}      \\
		\Luc{kn}={} & \dfrac{5\Fib{(k-1)n}\Fib{n}+\Luc{(k-1)n}\Luc{n}}{2} \\
	\end{align*}
\end{cor}
\begin{proof}
	\begin{align*}
		\intertext{Utilizzando il~\vref{thm:LucFibSommaprodotto}}
		\Fib{n+m}={} & \dfrac{\Fib{m}\Luc{n}+\Luc{m}\Fib{n}}{2}            \\
		\Luc{n+m}={} & \dfrac{5\Fib{m}\Fib{n}+\Luc{m}\Luc{n}}{2}
		\intertext{Ponendo}
		kn={}        & kn-n+n=n(k-1)+n                                     \\
		m={}         & n(k-1)                                              \\
		n={}         & n                                                   \\
		\Fib{kn}={}  & \dfrac{\Fib{n(k-1)}\Luc{n}+\Luc{n(k-1)}\Fib{n}}{2}  \\
		\Luc{kn}={}  & \dfrac{5\Fib{n(k-1)}\Fib{n}+\Luc{n(k-1)}\Luc{n}}{2}
	\end{align*}
	cvd.
\end{proof}
\begin{thm}[Relazione fondamentale fra Lucas e
		Fibonacci]\label{thm:FibLucFondamentale}~\cite{Rabinowitz_1996}
	Se $\Fib{n}$ è la successione di Fibonacci e  $\Luc{n}$ è quella di Lucas allora:
	\begin{equation}
		\Luc{n}^2-5\Fib{n}^2=4(-1)^{n}
	\end{equation}\label{eqn:FibLucFondamentale}
\end{thm}
\begin{proof}
	\begin{align*}
		\Luc{n}^2={}            & a^{2n}+2a^nb^n+b^{2n}                                                        \\
		\Fib{n}^2={}            & \dfrac{a^{2n}}{(a-b)^2}+\dfrac{2a^{n}b^{n}}{(a-b)^2}+\dfrac{b^{2n}}{(a-b)^2} \\
		\Luc{n}^2-5\Fib{n}^2={} & \dfrac{(a^2-2ab+b^2-5)}{(a-b)^2}                                             \\
		={}                     & \dfrac{(a^2-2ab+b^2-5)(a^{2n}+b^{2n})+2a^nb^n(a^2+2ab+b^2+5)}{(a-b)^2}       \\
		={}                     & \dfrac{[(a-b)^2-5](a^{2n}+b^{2n})+2a^nb^n[(a-b)^2+5]}{(a-b)^2}               \\
		\intertext{Per il~\vref{lem:FibpropPhi}}
		={}                     & \dfrac{2(-1)^n[5+5]}{5}                                                      \\
		={}                     & 4(-1)^n                                                                      \\
	\end{align*}
	Da cui la tesi.
\end{proof}
\section{Formule di conversione}
\begin{thm}[Fibonacci e Lucas
		negativo]~\cite{Rabinowitz_1996}\label{thm:FibLucNeg}
	Se $\Fib{n}$ è la successione di Fibonacci e  $\Luc{n}$ è quella di Lucas
	allora:
	\begin{align}
		\Fib{-n}={} & (-1)^{n+1}\Fib{n}\label{eqn:FibNegate} \\
		\Luc{-n}={} & (-1)^{n}\Luc{n}\label{eqn:LucNegate}
	\end{align}
\end{thm}
\begin{proof}
	\proofpart{Fibonacci}
	Riscriviamo il risultato dell'~\vref{eqn:FinBinet}
	\begin{align*}
		\Fib{n}={}  & \dfrac{1}{a-b}\left(a^n-b^n\right)       \\
		\Fib{-n}={} & \dfrac{1}{a-b}\left(a^{-n}-b^{-n}\right) \\
		={}         & \dfrac{(ab)^{-n}}{b-a}[a^n-b^n]
		\intertext{Per il~\vref{lem:FibpropPhi}}
		={}         & \dfrac{(-1)^{n+1}}{a-b}[a^n-b^n]         \\
		={}         & (-1)^{n+1}\Fib{n}
	\end{align*}
	\proofpart{Lucas}
	Riscriviamo il risultato dell'~\vref{eqn:LucBinet}
	\begin{align*}
		\Luc{n}={}  & a^{n}+b^{n}         \\
		\Luc{-n}={} & a^{-n}-b^{-n}       \\
		={}         & (ab)^{-n}[a^n-b^n]
		\intertext{Per il~\vref{lem:FibpropPhi}}
		={}         & (-1)^{n+1}[a^n-b^n] \\
		={}         & (-1)^{n}\Luc{n}
	\end{align*}
	cvd.
\end{proof}
\begin{thm}[Converti in Fibonacci o in
		Lucas]\label{thm:LucasToFibFibToLuc}~\cite{Rabinowitz_1996}
	Se $\Fib{n}$ è la successione di Fibonacci e  $\Luc{n}$ è quella di Lucas allora:
	\begin{align}
		\Luc{n}={} & \Fib{n+1}+\Fib{n-1}\label{eqn:LucasConvertiinFib}            \\
		\Fib{n}={} & \dfrac{\Luc{n+1}+\Luc{n-1}}{5}\label{eqn:FibConvertiinLucas}
	\end{align}
\end{thm}
\begin{proof}
	\proofpart{Fibonacci}
	\begin{align*}
		\intertext{Per il~\vref{thm:LucFibSommaprodotto}}
		\Fib{n+m}={}           & \dfrac{\Fib{m}\Luc{n}+\Luc{m}\Fib{n}}{2}                                 \\
		\Fib{n+1}={}           & \dfrac{\Fib{1}\Luc{n}+\Luc{1}\Fib{n}}{2}                                 \\
		\Fib{n-1}={}           & \dfrac{\Fib{-1}\Luc{n}+\Luc{-1}\Fib{n}}{2}                               \\
		\Fib{n+1}+\Fib{n-1}={} & \dfrac{\Fib{1}\Luc{n}+\Luc{1}\Fib{n}+\Fib{-1}\Luc{n}+\Luc{-1}\Fib{n}}{2} \\
		\intertext{Per il~\vref{thm:FibLucNeg}}
		\Fib{-1}={}            & (-1)^{-1+1}\Fib{1}=\Fib{1}                                               \\
		\Luc{-1}={}            & (-1)^{1}\Luc{1}=-	\Luc{1}
		\intertext{Quindi}
		={}                    & \dfrac{\Fib{1}\Luc{n}+\Luc{1}\Fib{n}+\Fib{1}\Luc{n}-\Luc{1}\Fib{n}}{2}   \\
		={}                    & \dfrac{\Fib{1}\Luc{n}+\Fib{1}\Luc{n}}{2}                                 \\
		={}                    & \Luc{n}
	\end{align*}
	\proofpart{Lucas}
	\begin{align*}
		\intertext{Per il~\vref{thm:LucFibSommaprodotto}}
		\Luc{n+m}={}           & \dfrac{5\Fib{m}\Fib{n}+\Luc{m}\Luc{n}}{2}                                  \\
		\Luc{n+1}={}           & \dfrac{5\Fib{1}\Fib{n}+\Luc{1}\Luc{n}}{2}                                  \\
		\Luc{n-1}={}           & \dfrac{5\Fib{-1}\Fib{n}+\Luc{-1}\Luc{n}}{2}                                \\
		\Luc{n+1}+\Luc{n-1}={} & \dfrac{5\Fib{1}\Fib{n}+\Luc{1}\Luc{n}+5\Fib{-1}\Fib{n}+\Luc{-1}\Luc{n}}{2} \\
		\intertext{Per il~\vref{thm:FibLucNeg}}
		\Fib{-1}={}            & (-1)^{-1+1}\Fib{1}=\Fib{1}                                                 \\
		\Luc{-1}={}            & (-1)^{1}\Luc{1}=-	\Luc{1}
		\intertext{Quindi}
		={}                    & \dfrac{5\Fib{1}\Fib{n}+\Luc{1}\Luc{n}+5\Fib{1}\Fib{n}-\Luc{1}\Luc{n}}{2}   \\
		={}                    & \dfrac{5\Fib{1}\Fib{n}+5\Fib{1}\Fib{n}}{2}                                 \\
		={}                    & \dfrac{10\Fib{1}\Fib{n}}{2}                                                \\
		={}                    & 5\Fib{n}
	\end{align*}
	Da cui la tesi.
\end{proof}
\begin{thm}[Prodotto in somma]~\cite{Rabinowitz_1996}\label{thm:FibProdSomma}
	Se $\Fib{n}$ è la successione di Fibonacci e  $\Luc{n}$ è quella di Lucas allora:
	\begin{align}
		\Fib{m}\Fib{n}={} & \dfrac{\Luc{n+m}-(-1)^n\Luc{m-n}}{5}\label{eqn:FibProdSomma} \\
		\Luc{m}\Luc{n}={} & \Luc{n+m}+(-1)^n\Luc{m-n}\label{eqn:LucProdSomma}            \\
		\Fib{m}\Luc{n}={} & \Fib{n+m}+(-1)^n\Fib{m-n}\label{eqn:LucFibProdSomma}
	\end{align}
\end{thm}
\begin{proof}
	\proofpart{Fibonacci}
	\begin{align*}
		\Fib{m}\Fib{n}={} & \dfrac{a^{m+n}}{(a-b)^2}-\dfrac{a^mb^n}{(a-b)^2}-\dfrac{a^nb^m}{(a-b)^2}+\dfrac{b^{m+n}}{(a-b)^2} \\
		={}               & \dfrac{a^{m+n}}{(a-b)^2}+\dfrac{b^{m+n}}{(a-b)^2}-\dfrac{a^mb^n}{(a-b)^2}-\dfrac{a^nb^m}{(a-b)^2} \\
		={}               & \dfrac{\Luc{m+n}}{(a-b)^2}-\dfrac{a^mb^n+a^nb^m}{(a-b)^2}                                         \\
		\Luc{m-n}={}      & a^{m-n}-b^{m-n}=a^{-n}b^{-n}(a^mb^n+a^nb^m)                                                       \\
		\intertext{Ma}
		a^mb^n+a^nb^m={}  & a^{n}b^{n}\Luc{m-n}                                                                               \\
		\intertext{Quindi}
		={}               & \dfrac{\Luc{m+n}}{(a-b)^2}-\dfrac{a^{n}b^{n}\Luc{m-n}}{(a-b)^2}                                   \\
		\intertext{Per il~\vref{lem:FibpropPhi}}
		={}               & \dfrac{\Luc{m+n}+(-1)^{n}\Luc{m-n}}{5}
	\end{align*}
	\proofpart{Lucas}
	\begin{align*}
		\Luc{m}\Luc{n}={} & a^{m+n}+a^mb^n+a^nb^m+b^{m+n}               \\
		={}               & a^{m+n}+b^{m+n}+a^mb^n+a^nb^m               \\
		={}               & \Luc{m+n}+(a^mb^n+a^nb^m)                   \\
		\Luc{m-n}={}      & a^{m-n}-b^{m-n}=a^{-n}b^{-n}(a^mb^n+a^nb^m) \\
		\intertext{Ma}
		a^mb^n+a^nb^m={}  & a^{n}b^{n}\Luc{m-n}                         \\
		\intertext{Quindi}
		={}               & \Luc{m+n}+a^{n}b^{n}\Luc{m-n}               \\
		\intertext{Per il~\vref{lem:FibpropPhi}}
		={}               & \Luc{m+n}+(-1)^{n}\Luc{m-n}
	\end{align*}
	\proofpart{Fibonacci Lucas}
	\begin{align*}
		\Fib{m}\Luc{n}={}             & \dfrac{a^{m+n}}{a-b}+\dfrac{a^mb^n}{a-b}+\dfrac{a^nb^m}{b-a}+\dfrac{b^{m+n}}{b-a}             \\
		={}                           & \dfrac{a^{m+n}}{a-b}+\dfrac{b^{m+n}}{b-a}+\dfrac{a^mb^n}{a-b}+\dfrac{a^nb^m}{b-a}             \\
		={}                           & \Fib{m+n}+\dfrac{a^mb^n-a^nb^m}{a-b}                                                          \\
		\Fib{m-n}={}                  & \dfrac{a^{m-n}}{a-b}+\dfrac{b^{m-n}}{b-a}=a^{-n}b^{-n}\left(\dfrac{a^mb^n+a^nb^m}{a-b}\right) \\
		\intertext{Ma}
		\dfrac{a^mb^n+a^nb^m}{a-b}={} & a^{n}b^{n}\Fib{m-n}                                                                           \\
		\intertext{Quindi}
		={}                           & \Fib{m+n}+a^{n}b^{n}\Fib{m-n}                                                                 \\
		\intertext{Per il~\vref{lem:FibpropPhi}}
		={}                           & \Fib{m+n}+(-1)^{n}\Fib{m-n}
	\end{align*}
	cvd.
\end{proof}
Conseguenza del~\vref{thm:FibProdSomma} è il seguente
\begin{cor}[Potenze in somma]~\cite{Rabinowitz_1996}\label{cor:FibpotSomma}
	Se $\Fib{n}$ è la successione di Fibonacci e  $\Luc{n}$ è quella di Lucas allora:
	\begin{align}
		\Fib{n}^2={} & \dfrac{\Luc{2n}-2(-1)^n}{5}\label{eqn:FibQuadSomma} \\
		\Luc{n}^2={} & \Luc{2n}+2(-1)^n\label{eqn:LucQuadSomma}
	\end{align}
\end{cor}
\begin{proof}
	\proofpart{Fibonacci}
	\begin{align*}
		\intertext{Per il~\vref{thm:FibProdSomma}}
		\Fib{m}\Fib{n}={} & \dfrac{\Luc{n+m}-(-1)^n\Luc{m-n}}{5}
		\intertext{posto}
		m={}              & n                                    \\
		\Fib{n}\Fib{n}={} & \dfrac{\Luc{n+n}-(-1)^n\Luc{n-n}}{5} \\
		\Fib{n}^2={}      & \dfrac{\Luc{2n}-(-1)^n\Luc{0}}{5}    \\
		\Fib{n}^2={}      & \dfrac{\Luc{2n}-2(-1)^n}{5}          \\
	\end{align*}
	\proofpart{Lucas }
	\begin{align*}
		\intertext{Per il~\vref{thm:FibProdSomma}}
		\Luc{m}\Luc{n}={} & \Luc{n+m}+(-1)^n\Luc{m-n}
		\intertext{posto}
		m={}              & n                         \\
		\Luc{n}\Luc{n}={} & \Luc{n+n}+(-1)^n\Luc{n-n} \\
		\Luc{n}^2={}      & \Luc{2n}+(-1)^n\Luc{0}    \\
		\Luc{n}^2={}      & \Luc{2n}+2(-1)^n          \\
	\end{align*}
	cvd.
\end{proof}
Simili al precedente sono  i seguenti risultati
\begin{thm}[Potenze in somma]~\cite{Rabinowitz_1996}\label{thm:FibpotSommadue}
	Se $\Fib{n}$ è la successione di Fibonacci e  $\Luc{n}$ è quella di Lucas allora:
	\begin{align}
		\Fib{n}^3={} & \dfrac{\Fib{3n}-3(-1)^n\Fib{n}}{5}\label{eqn:FibCubSomma}                           \\
		\Luc{n}^3={} & \Luc{3n}+3(-1)^n\Luc{n}\label{eqn:LucCubSomma}                                      \\
		\Fib{n}^4={} & \dfrac{\Luc{4n}-4(-1)^n\Luc{2n}+6}{25}\label{eqn:FibQuartaSomma}                    \\
		\Luc{n}^4={} & \Luc{4n}+4(-1)^n\Luc{2n}+6\label{eqn:LucQuartaSomma}                                \\
		\Fib{n}^5={} & \dfrac{\Fib{5n}-5(-1)^n\Fib{3n}+10\Fib{n}}{25}\label{eqn:FibQuintaSomma}            \\
		\Luc{n}^5={} & \Luc{5n}+5(-1)^n\Luc{3n}+10\Luc{n}\label{eqn:LucQuintaSomma}                        \\
		\Fib{n}^6={} & \dfrac{\Luc{6n}-6(-1)^n\Luc{4n}+15\Luc{2n}-20(-1)^n}{125}\label{eqn:FibSestsaSomma} \\
		\Luc{n}^6={} & \Luc{6n}+6(-1)^n\Luc{4n}+15\Luc{2n}+20(-1)^n\label{eqn:LucSestsaSomma}
	\end{align}
\end{thm}
\begin{proof}
	\proofpart{Cubo Fibonacci}
	\begin{align*}
		\Fib{n}^3={}       & \Fib{n}^2\Fib{n}                                    \\
		\intertext{Per il~\vref{cor:FibpotSomma}}
		={}                & \dfrac{\Luc{2n}-2(-1)^n}{5}\Fib{n}                  \\
		={}                & \dfrac{\Luc{2n}\Fib{n}-2(-1)^n\Fib{n}}{5}           \\
		\intertext{Per il~\vref{thm:FibProdSomma}}
		\Fib{m}\Luc{n}={}  & \Fib{n+m}+(-1)^n\Fib{m-n}
		\intertext{Quindi}
		\Fib{n}\Luc{2n}={} & \Fib{n+2n}+(-1)^{2n}\Fib{n-2n}                      \\
		={}                & \Fib{3n}+\Fib{-n}                                   \\
		\intertext{Per il~\vref{thm:FibLucNeg}}
		={}                & \Fib{3n}+(-1)^{n+1}\Fib{-n}                         \\
		={}                & \Fib{3n}-(-1)^{n}\Fib{n}                            \\
		\intertext{Quindi}
		={}                & \dfrac{\Fib{3n}-(-1)^{n}\Fib{n} -2(-1)^n\Fib{n}}{5} \\
		={}                & \dfrac{\Fib{3n}-3(-1)^n\Fib{n}}{5}                  \\
	\end{align*}
	\proofpart{Cubo Lucas}
	\begin{align*}
		\Luc{n}^3={}       & \Luc{n}^2\Luc{n}                      \\
		\intertext{Per il~\vref{cor:FibpotSomma}}
		={}                & (\Luc{2n}+2(-1)^n)\Luc{n}             \\
		={}                & \Luc{2n}\Luc{n}+2(-1)^n\Luc{n}        \\
		\intertext{Per il~\vref{thm:FibProdSomma}}
		\Luc{2n}\Luc{n}={} & \Luc{2n+n}+(-1)^n\Luc{2n-n}           \\
		={}                & \Luc{3n}+(-1)^n\Luc{n}                \\
		\intertext{Quindi}
		={}                & \Luc{3n}+(-1)^n\Luc{n}+2(-1)^n\Luc{n} \\
		={}                & \Luc{3n}+3(-1)^n\Luc{n}               \\
	\end{align*}
	\proofpart{Fibonacci Quarta}
	\begin{align*}
		\Fib{n}^4={}       & \Fib{n}^3\Fib{n}                                               \\
		\intertext{Per l'~\vref{eqn:FibCubSomma}}
		={}                & \dfrac{\Fib{3n}-3(-1)^n\Fib{n}}{5}\Fib{n}                      \\
		={}                & \dfrac{\Fib{3n}\Fib{n}-3(-1)^n\Fib{n}^2}{5}                    \\
		\intertext{Per il~\vref{thm:FibProdSomma}}
		\Fib{3n}\Fib{n}={} & \dfrac{\Luc{3n+n}-(-1)^n\Luc{3n-n}}{5}                         \\
		\Fib{3n}\Fib{n}={} & \dfrac{\Luc{4n}-(-1)^n\Luc{2n}}{5}                             \\
		\Fib{n}^2={}       & \dfrac{\Luc{2n}-2(-1)^n}{5}                                    \\
		\intertext{Quindi}
		={}                & \dfrac{\Luc{4n}-(-1)^n\Luc{2n}-3(-1)^n[\Luc{2n}-2(-1)^n]}{25}  \\
		={}                & \dfrac{\Luc{4n}-(-1)^n\Luc{2n}-3(-1)^n\Luc{2n}+6(-1)^{2n}}{25} \\
		={}                & \dfrac{\Luc{4n}-4(-1)^n\Luc{2n}+6}{25}                         \\
	\end{align*}
	\proofpart{Lucas Quarta}
	\begin{align*}
		\Luc{n}^4={}       & \Luc{n}^3\Luc{n}                                   \\
		\intertext{Per l'~\vref{eqn:LucCubSomma}}
		={}                & [\Luc{3n}+3(-1)^n\Luc{n}]\Luc{n}                   \\
		={}                & \Luc{3n}\Luc{n}+3(-1)^n\Luc{n}^2                   \\
		\intertext{Per il~\vref{thm:FibProdSomma}}
		\Luc{3n}\Luc{n}={} & \Luc{3n+n}+(-1)^n\Luc{3n-n}                        \\
		={}                & \Luc{4n}+(-1)^n\Luc{2n}                            \\
		\Luc{n}^2={}       & \Luc{2n}+2(-1)^n                                   \\
		\intertext{Quindi}
		={}                & \Luc{4n}+(-1)^n\Luc{2n}+3(-1)^n[\Luc{2n}+2(-1)^n]  \\
		={}                & \Luc{4n}+(-1)^n\Luc{2n}+3(-1)^n\Luc{2n}+6(-1)^{2n} \\
		={}                & \Luc{4n}+4(-1)^n\Luc{2n}+6                         \\
	\end{align*}
	\proofpart{Fibonacci Quinta}
	\begin{align*}
		\Fib{n}^5={}       & \Fib{n}^4\Fib{n}                                                                      \\
		\intertext{Per l'~\vref{eqn:FibQuartaSomma}}
		={}                & \dfrac{\Luc{4n}-4(-1)^n\Luc{2n}+6}{25}\Fib{n}                                         \\
		={}                & \dfrac{\Luc{4n}\Fib{n}-4(-1)^n\Luc{2n}\Fib{n}+6\Fib{n}}{25}                           \\
		\intertext{Per il~\vref{thm:FibProdSomma}}
		\Fib{m}\Luc{n}={}  & \Fib{n+m}+(-1)^n\Fib{m-n}                                                             \\
		\Fib{n}\Luc{4n}={} & \Fib{4n+n}+(-1)^{4n}\Fib{n-4n}                                                        \\
		={}                & \Fib{5n}+\Fib{-3n}                                                                    \\
		={}                & \Fib{5n}+(-1)^{3n+1}\Fib{3n}                                                          \\
		\Fib{n}\Luc{2n}={} & \Fib{2n+n}+(-1)^{2n}\Fib{n-2n}                                                        \\
		={}                & \Fib{3n}+(-1)^{2n}\Fib{n-2n}                                                          \\
		={}                & \Fib{3n}+\Fib{-n}                                                                     \\
		={}                & \Fib{3n}+(-1)^{n+1}\Fib{n}                                                            \\
		\intertext{Quindi}
		={}                & \dfrac{\Fib{5n}+(-1)^{3n+1}\Fib{3n}-4(-1)^n[\Fib{3n}+(-1)^{n+1}\Fib{n}]+6\Fib{n}}{25} \\
		={}                & \dfrac{\Fib{5n}+(-1)^{3n+1}\Fib{3n}-4(-1)^n\Fib{3n}-4(-1)^{2n+1}\Fib{n}+6\Fib{n}}{25} \\
		={}                & \dfrac{\Fib{5n}-(-1)^{n}\Fib{3n}-4(-1)^n\Fib{3n}-4(-1)^{1}\Fib{n}+6\Fib{n}}{25}       \\
		={}                & \dfrac{\Fib{5n}-5(-1)^n\Fib{3n}+10\Fib{n}}{25}                                        \\
	\end{align*}
	\proofpart{Lucas Quinta}
	\begin{align*}
		\Luc{n}^5={}       & \Luc{n}^4\Luc{n}                                                     \\
		\intertext{Per l'~\vref{eqn:LucQuartaSomma}}
		={}                & (\Luc{4n}+4(-1)^n\Luc{2n}+6)\Luc{n}                                  \\
		={}                & \Luc{4n}\Luc{n}+4(-1)^n\Luc{2n}\Luc{n}+6\Luc{n}                      \\
		\intertext{Per il~\vref{thm:FibProdSomma}}
		\Luc{m}\Luc{n}={}  & \Luc{m+n}+(-1)^{n}\Luc{m-n}                                          \\
		\Luc{4n}\Luc{n}={} & \Luc{4n+n}+(-1)^{n}\Luc{4n-n}                                        \\
		={}                & \Luc{5n}+(-1)^{n}\Luc{3n}                                            \\
		\Luc{2n}\Luc{n}={} & \Luc{2n+n}+(-1)^{n}\Luc{2n-n}                                        \\
		={}                & \Luc{3n}+(-1)^{n}\Luc{n}                                             \\
		\intertext{Quindi}
		={}                & \Luc{5n}+(-1)^{n}\Luc{3n}+4(-1)^n[\Luc{3n}+(-1)^{n}\Luc{n}]+6\Luc{n} \\
		={}                & \Luc{5n}+(-1)^{n}\Luc{3n}+4(-1)^n\Luc{3n}+4(-1)^{2n}\Luc{n}+6\Luc{n} \\
		={}                & \Luc{5n}+5(-1)^n\Luc{3n}+4\Luc{n}+6\Luc{n}                           \\
		={}                & \Luc{5n}+5(-1)^n\Luc{3n}+10\Luc{n}                                   \\
	\end{align*}
	\proofpart{Fibonacci Sesta}
	\begin{align*}
		\Fib{n}^6={}       & \Fib{n}^4\Fib{n}                                                                           \\
		\intertext{Per l'~\vref{eqn:FibQuintaSomma}}
		={}                & \dfrac{\Fib{5n}\Fib{n}-5(-1)^n\Fib{3n}\Fib{n}+10\Fib{n}^2}{25}                             \\
		\Fib{m}\Fib{n}={}  & \dfrac{\Luc{n+m}-(-1)^n\Luc{m-n}}{5}                                                       \\
		\Fib{5m}\Fib{n}={} & \dfrac{\Luc{5n+n}-(-1)^n\Luc{5n-n}}{5}                                                     \\
		={}                & \dfrac{\Luc{6n}-(-1)^n\Luc{4n}}{5}                                                         \\
		\Fib{3n}\Fib{n}={} & \dfrac{\Luc{3n+n}-(-1)^n\Luc{3n-n}}{5}                                                     \\
		={}                & \dfrac{\Luc{4n}-(-1)^n\Luc{2n}}{5}                                                         \\
		\Luc{n}^2={}       & \Luc{2n}+2(-1)^n                                                                           \\
		\intertext{Quindi}
		={}                & \dfrac{\Luc{6n}-(-1)^n\Luc{4n}-5(-1)^n\Luc{4n}+5(-1)^{2n}\Luc{2n}+10\Luc{2n}+20(-1)^n}{25} \\
		={}                & \dfrac{\Luc{6n}-6(-1)^n\Luc{4n}+15(-1)^{2n}\Luc{2n}+20(-1)^n}{25}                          \\
	\end{align*}
	\proofpart{Lucas Sesta}
	\begin{align*}
		\Luc{n}^6={}       & \Luc{n}^5\Luc{n}                                                                   \\
		\intertext{Per l'~\vref{eqn:LucQuintaSomma}}
		={}                & [\Luc{5n}+5(-1)^n\Luc{3n}+10\Luc{n}]\Luc{n}                                        \\
		={}                & \Luc{5n}\Luc{n}+5(-1)^n\Luc{3n}\Luc{n}+10\Luc{n}^2                                 \\
		\intertext{Per il~\vref{thm:FibProdSomma}}
		\Luc{m}\Luc{n}={}  & \Luc{m+n}+(-1)^{n}\Luc{m-n}                                                        \\
		\Luc{5n}\Luc{n}={} & \Luc{5n+n}+(-1)^{n}\Luc{5n-n}                                                      \\
		={}                & \Luc{4n}+(-1)^{n}\Luc{4n}                                                          \\
		\Luc{3n}\Luc{n}={} & \Luc{3n+n}+(-1)^{n}\Luc{3n-n}                                                      \\
		={}                & \Luc{4n}+(-1)^{n}\Luc{2n}                                                          \\
		\Luc{n}^2={}       & \Luc{2n}+2(-1)^n
		\intertext{Quindi}
		={}                & \Luc{5n}+(-1)^{n}\Luc{4n}+5(-1)^n[\Luc{4n}+(-1)^{n}\Luc{2n}]+10[\Luc{4n}+(-1)^{n}] \\
		={}                & \Luc{6n}+(-1)^{n}\Luc{4n}+5(-1)^n\Luc{4n}+5(-1)^{2n}\Luc{2n}+10\Luc{4n}+20(-1)^{n} \\
		={}                & \Luc{6n}+6(-1)^{n}\Luc{4n}+15\Luc{2n}+20(-1)^{n}                                   \\
	\end{align*}
	cvd.
\end{proof}
\section{Radici}
\begin{thm}[Algoritmo per rimuovere a e
		b]~\cite{Rabinowitz_1996}\label{thm:FibLucRimuoviab}
	Se $\Fib{n}$ è la successione di Fibonacci e  $\Luc{n}$ è quella di Lucas
	allora:\begin{equation}
		\left\{\begin{aligned}
			a^n={} & \dfrac{\Luc{n}+(a-b)\Fib{n}}{2} \\
			b^n={} & \dfrac{\Luc{n}-(a-b)\Fib{n}}{2} \\
		\end{aligned}
		\right.
	\end{equation}
\end{thm}
\begin{proof}
	Utilizzando il~\vref{thm:FibFormulaBinet} e il~\vref{thm:LucFormulaBinet}
	possiamo scrivere
	\begin{equation*}
		\left\{
		\begin{aligned}
			\Fib{n}={} & \dfrac{1}{a-b}\left(a^n-b^n\right) \\
			\Luc{n}={} & a^n+b^n
		\end{aligned}
		\right.
	\end{equation*}
	che risolto rispetto $a^n$ e $b^n$ da la tesi.
\end{proof}
\begin{thm}[Potenze di a e 	b]~\cite{Rabinowitz_1996}
	\label{thm:FibRimuovipotab
	}
	Se $\Fib{n}$ è la successione di Fibonacci allora:
	\begin{align*}
		a^n={} & a\Fib{n}+\Fib{n-1} \\
		b^n={} & b\Fib{n}+\Fib{n-1} \\
	\end{align*}
\end{thm}
\begin{proof}
	\proofpart{Prima dimostrazione}
	Utilizzando il~\vref{thm:FibFormulaBinet} possiamo scrivere:
	\begin{align*}
		a\Fib{n}+\Fib{n-1}={} &                                         \\
		={}                   & a\dfrac{1}{a-b}\left(a^n-b^n\right)
		+\dfrac{1}{a-b}\left(a^{n-1}-b^{n-1}\right)                     \\
		={}                   & \dfrac{ba^n(a^2+1)-ab^n(ab+1)}{ab(a-b)} \\
		={}                   & \dfrac{a^{n-1}(a^2+1)}{a-b}             \\
		={}                   & \dfrac{a^{n-1}a(a-b)}{a-b}              \\
		={}                   & a^n                                     \\
		b\Fib{n}+\Fib{n-1}={} &                                         \\
		={}                   & b\dfrac{1}{a-b}\left(a^n-b^n\right)
		+\dfrac{1}{a-b}\left(a^{n-1}-b^{n-1}\right)                     \\
		={}                   & \dfrac{ba^n(ab+1)-ab^n(b^2+1)}{ab(a-b)} \\
		={}                   & \dfrac{b^{n-1}(b^2+1)}{b-a}             \\
		={}                   & \dfrac{b^{n-1}b(b-a)}{b-a}              \\
		={}                   & b^n
	\end{align*}
	\proofpart{Seconda dimostrazione}

	Poniamo che
	\begin{align*}
		       & \left\{
		\begin{aligned}
			a^1={} & x\Fib{1}+y\Fib{0} \\
			a^2={} & x\Fib{2}+y\Fib{1} \\
		\end{aligned}
		\right.                            \\
		       & \left\{
		\begin{aligned}
			x= & a     \\
			y= & a^2-a
		\end{aligned}
		\right.                            \\
		a^n={} & a\Fib{n}+(a^2-a)\Fib{n-1} \\
		={}    & a\Fib{n}+\Fib{n-1}        \\
	\end{align*}
	Analogamente l'altra relazione.

	Cvd.
\end{proof}
\section{Formule di sottrazione}
\begin{thm}[Formule di
		sottrazione]~\cite{Rabinowitz_1996}\label{thm:FibLucFormSottazione}
	Se $\Fib{n}$ è la successione di Fibonacci e  $\Luc{n}$ è quella di Lucas
	allora:
	\begin{align}
		\Fib{m-n}={} & (-1)^n\dfrac{\Fib{m}\Luc{n}-\Luc{m}\Fib{n}}{2}  \\
		\Luc{m-n}={} & (-1)^n\dfrac{\Luc{m}\Luc{n}-5\Fib{m}\Fib{n}}{2}
	\end{align}
\end{thm}
\begin{proof}
	\proofpart{Fibonacci}
	Utilizzando l'~\vref{eqn:FibLucSommaprodotto}
	\begin{align*}
		\Fib{m+n}={} & \dfrac{\Fib{m}\Luc{n}+\Luc{m}\Fib{n}}{2}                 \\
		\intertext{Posto}
		n={}         & -n                                                       \\
		\Fib{m-n}={} & \dfrac{\Fib{m}\Luc{-n}+\Luc{m}\Fib{-n}}{2}               \\
		\intertext{Utilizzando il~\vref{thm:FibLucNeg} }
		={}          & \dfrac{\Fib{m}(-1)^n\Luc{n}+\Luc{m}(-1)^{n+1}\Fib{n}}{2} \\
		={}          & (-1)^n\dfrac{\Fib{m}\Luc{n}-\Luc{m}\Fib{n}}{2}
	\end{align*}
	\proofpart{Lucas}
	Utilizzando l'~\vref{eqn:LucFibSommaprodotto}
	\begin{align*}
		\Luc{m+n}={} & \dfrac{5\Fib{m}\Fib{n}+\Luc{m}\Luc{n}}{2}                 \\
		\intertext{Posto}
		n={}         & -n                                                        \\
		\Luc{m-n}={} & \dfrac{5\Fib{m}\Fib{-n}+\Luc{m}\Luc{-n}}{2}               \\
		\intertext{Utilizzando il~\vref{thm:FibLucNeg} }
		={}          & \dfrac{5\Fib{m}\Fib{n}(-1)^{n+1}+\Luc{m}(-1)^n\Luc{n}}{2} \\
		={}          & (\Luc{m}\Luc{n}-1^n)\dfrac{-5\Fib{m}\Fib{n}}{2}           \\
	\end{align*}
	cvd.
\end{proof}
\section{Limiti}
\begin{thm}[Limite successione]
	Se $\Fib{n}$ è la successione di Fibonacci allora
	\begin{equation}
		\lim_{n\to\infty}\dfrac{\Fib{n+1}}{\Fib{n}}=\varphi
	\end{equation}\label{eqn:FibLimRap}
\end{thm}
\begin{proof}
	\begin{align*}
		\lim_{n\to\infty}\dfrac{\Fib{n+1}}{\Fib{n}}={}                  & \lim_{n\to\infty}\dfrac{\varphi^{n+1}-(1-\varphi)^{n+1}}{\varphi^n-(1-\varphi)^n}                                                                                         \\
		={}                                                             & \lim_{n\to\infty}\dfrac{\varphi^{n+1}\left[1-\left(\dfrac{1-\varphi}{\varphi}\right)^{n+1}\right]}{\varphi^{n}\left[1-\left(\dfrac{1-\varphi}{\varphi}\right)^{n}\right]} \\
		={}                                                             & \lim_{n\to\infty}\dfrac{\varphi\left[1-\left(\dfrac{1-\varphi}{\varphi}\right)^{n+1}\right]}{1-\left(\dfrac{1-\varphi}{\varphi}\right)^{n}}                               \\
		={}                                                             & \varphi
		\intertext{dato che}
		\lim_{n\to\infty}\left(\dfrac{1-\varphi}{\varphi}\right)^{n}={} & 0
	\end{align*}
	cvd.
\end{proof}
\section{Identità}
\begin{thm}[Identità di
		Cassini]\label{thm:FibCassini}\index{Fibonacci!identità!Cassini}\index{Cassini}
	Se $\Fib{n}$ è la successione di Fibonacci allora
	\begin{equation}
		\Fib{n-1}\Fib{n+1}=\Fib{n}^2+(-1)^n
	\end{equation}\label{eqn:FibCassini}
\end{thm}
\begin{proof}

	\proofpart{Prima dimostrazione}
	Riscriviamo il risultato dell'~\vref{eqn:FinBinet}
	\begin{align*}
		\Fib{n}={}            & \dfrac{1}{a-b}\left(a^n-b^n\right)
		\intertext{da cui}
		\Fib{n}^2={}          & \dfrac{a^2n}{(a-b)^2}-2\dfrac{a^nb^n}{(a-b)^n}+\dfrac{b^2n}{(a-b)^2}                       \\
		\intertext{Ma}
		\Fib{n-1}={}          & \dfrac{1}{a-b}\left(a^{n-1}-b^{n-1}\right)                                                 \\
		\Fib{n+1}={}          & \dfrac{1}{a-b}\left(a^{n+1}-b^{n+1}\right)                                                 \\
		\intertext{Quindi}
		\Fib{n-1}\Fib{n+1}={} & \dfrac{a^2n}{(a-b)^2}-2\dfrac{a^nb^n}{(a-b)^n}+\dfrac{b^2n}{(a-b)^2}-a^{n-1}b^{n-1}        \\
		={}                   & \Fib{n}^2-a^{n-1}b^{n-1}
		\intertext{Ma}
		-a^{n-1}b^{n-1}={}    & -\left(\dfrac{1+\sqrt{5}}{2}\right)^{n-1}\left(\dfrac{1-\sqrt{5}}{2}\right)^{n-1}=(-1)^{n} \\
		\Fib{n-1}\Fib{n+1}={} & \Fib{n}^2+(-1)^n
	\end{align*}
	cvd.
	\proofpart{Seconda dimostrazione}
	\begin{align*}
		\Fib{n-1}\Fib{n+1}={} &                                               \\
		\intertext{Per il~\vref{thm:LucFibSommaprodotto}}
		\Fib{n+m}={}          & \dfrac{\Fib{m}\Luc{n}+\Luc{m}\Fib{n}}{2}      \\
		\Fib{n-1}={}          & \dfrac{\Fib{-1}\Luc{n}+\Luc{-1}\Fib{n}}{2}    \\
		\intertext{Per il~\vref{thm:FibLucNeg}}
		\Fib{-1}={}           & (-1)^{-1+1}\Fib{1}=\Fib{1}                    \\
		\Luc{-1}={}           & (-1)^{1}\Luc{1}=-	\Luc{1}                     \\
		={}                   & \dfrac{\Fib{1}\Luc{n}-\Luc{1}\Fib{n}}{2}      \\
		={}                   & \dfrac{\Luc{n}-\Fib{n}}{2}                    \\
		\Fib{n+1}={}          & \dfrac{\Luc{n}+\Fib{n}}{2}                    \\
		\intertext{Quindi}
		={}                   & \dfrac{(\Luc{n}-\Fib{n})(\Luc{n}+\Fib{n})}{2} \\
		={}                   & \dfrac{\Luc{n}^2-\Fib{n}^2}{4}                \\
		\intertext{Per il~\vref{thm:FibLucFondamentale}}
		={}                   & \dfrac{5\Fib{n}^2+4(-1)^{n}-\Fib{n}^2}{4}     \\
		={}                   & \Fib{n}^2+(-1)^{n}
	\end{align*}
	cvd.
\end{proof}

\begin{thm}[Identità di Catalan]
	Se $\Fib{n}$ è la successione di Fibonacci allora
	\begin{equation}
		\Fib{n}^2-\Fib{n-r}\Fib{n+r}=(-1)^{n-r}\Fib{r}^2
	\end{equation}\label{eqn:FibCatalan}
\end{thm}\index{Fibonacci!identità!Catalan}\index{Catalan}
\begin{proof}

	\proofpart{Prima dimostrazione}
	Riscriviamo il risultato dell'~\vref{eqn:FinBinet}
	\begin{align*}
		\Fib{n}={}            & \dfrac{1}{a-b}\left(a^n-b^n\right)
		\intertext{da cui}
		\Fib{n}^2={}          & \dfrac{a^{2n}-2a^nb^n+b^{2n}}{(a-b)^2}                                                       \\\Fib{n-r}={}&\dfrac{1}{a-b}\left(a^{n-r}-b^{n-r}\right)\\
		\Fib{n+r}={}          & \dfrac{1}{a-b}\left(a^{n+r}-b^{n+r}\right)                                                   \\
		\Fib{n}^2-
		\Fib{n-r}\Fib{n+r}={} & \dfrac{a^{n+r}b^{n-r}}{(a-b)^2}+\dfrac{a^{n-r}b^{n+r}}{(a-b)^2}-\dfrac{2a^{n}b^{n}}{(a-b)^2} \\
		={}                   & \dfrac{a^{n-r}b^{n-r}(a^r-b^r)^2}{(a-b)^2}
		\intertext{Per~\vref{lem:FibpropPhi}}
		={}                   & \dfrac{(-1)^{n-r}(a^r-b^r)^2}{(a-b)^2}                                                       \\
		={}                   & (-1)^{n-r}\Fib{r}^2                                                                          \\
	\end{align*}
	cvd

	\proofpart{Seconda dimostrazione}
	\begin{align*}
		\intertext{Per~\vref{thm:FibLucFormSottazione}}
		\Fib{n-r}={}                                                           & (-1)^r\dfrac{\Fib{n}\Luc{r}-\Luc{n}\Fib{r}}{2}                                                        \\
		\intertext{Per~\vref{thm:LucFibSommaprodotto}}
		\Fib{n+r}={}                                                           & \dfrac{\Fib{n}\Luc{r}+\Luc{n}\Fib{r}}{2}                                                              \\
		\Fib{n-r}\Fib{n+r}
		={}                                                                    & \dfrac{(-1)^r}{4}\left(\Fib{n}\Luc{r}-\Luc{n}\Fib{r}\right)\left(\Fib{n}\Luc{r}+\Luc{n}\Fib{r}\right) \\
		={}                                                                    & \dfrac{(-1)^r}{4}\left(\Fib{n}^2\Luc{r}^2-\Luc{n}^2\Fib{r}^2\right)                                   \\
		\intertext{Per il~\vref{cor:FibpotSomma}}
		\Fib{n}^2={}                                                           & \dfrac{\Luc{2n}-2(-1)^n}{5}                                                                           \\
		\Fib{r}^2={}                                                           & \dfrac{\Luc{2r}-2(-1)^r}{5}                                                                           \\
		\Luc{n}^2={}                                                           & \Luc{2n}+2(-1)^n                                                                                      \\
		\Luc{r}^2={}                                                           & \Luc{2r}+2(-1)^r                                                                                      \\
		\Fib{n}^2\Luc{r}^2={}                                                  & \dfrac{1}{5}\left(\Luc{2n}-2(-1)^n\right)\left(\Luc{2r}+2(-1)^r\right)                                \\
		={}                                                                    & \dfrac{1}{5}[\Luc{2n}\Luc{2r}+2(-1)^r\Luc{2n}-2(-1)^n\Luc{2r}-4(-1)^{n+r}]                            \\
		\Luc{n}^2\Fib{r}^2={}                                                  & \dfrac{1}{5}\left(\Luc{2n}+2(-1)^n\right)\left(\Luc{2r}-2(-1)^r
		\right)                                                                                                                                                                        \\
		={}                                                                    & \dfrac{1}{5}[\Luc{2n}\Luc{2r}-2(-1)^r\Luc{2n}+2(-1)^n\Luc{2r}-4(-1)^{n+r}]                            \\
		\intertext{Quindi}
		\dfrac{(-1)^r}{4}\left(\Fib{n}^2\Luc{r}^2-\Luc{n}^2\Fib{r}^2\right)={} &                                                                                                       \\
		={}                                                                    &
		\dfrac{(-1)^r}{4}\dfrac{1}{5}[\Luc{2n}\Luc{2r}+2(-1)^r\Luc{2n}-2(-1)^n\Luc{2r}-4(-1)^{n+r}                                                                                     \\
		-                                                                      & \Luc{2n}\Luc{2r}+2(-1)^r\Luc{2n}-2(-1)^n\Luc{2r}+4(-1)^{n+r}]                                         \\
		={}                                                                    & \dfrac{(-1)^r}{4}\dfrac{1}{5}[4(-1)^r\Luc{2n}-4(-1)^n\Luc{2r}]                                        \\
		={}                                                                    & \dfrac{(-1)^r}{5}[(-1)^r\Luc{2n}-(-1)^n\Luc{2r}]                                                      \\
		\intertext{Ricapitolando}
		\Fib{n}^2-\Fib{n-r}\Fib{n+r}={}                                        &
		\intertext{Per il~\vref{cor:FibpotSomma}}
		={}                                                                    & \dfrac{\Luc{2n}-2(-1)^n}{5}-\dfrac{(-1)^r}{5}[(-1)^r\Luc{2n}-(-1)^n\Luc{2r}]                          \\
		={}                                                                    & \dfrac{\Luc{2n}-2(-1)^n- (-1)^{2r}\Luc{2n}+(-1)^{n+r}\Luc{2r}}{5}                                     \\
		={}                                                                    & \dfrac{-2(-1)^n+(-1)^{n+r}\Luc{2r}}{5}                                                                \\
		={}                                                                    & (-1)^{n-r}\dfrac{-2(-1)^r+(-1)^{2r}\Luc{2r}}{5}                                                       \\
		={}                                                                    & (-1)^{n-r}\dfrac{-2(-1)^r+\Luc{2r}}{5}                                                                \\
		\intertext{Per~\vref{lem:FibpropPhi}}
		={}                                                                    & (-1)^{n-r}\Fib{r}^2                                                                                   \\
	\end{align*}
	cvd.
\end{proof}
\begin{thm}[Identità di Vajada]\label{thm:FibVajada}
	Se $\Fib{n}$ è la successione di Fibonacci allora
	\begin{equation}
		\Fib{n+i}\Fib{n+j}-\Fib{n}\Fib{n+i+j}=(-1)^{n}\Fib{i}\Fib{j}
	\end{equation}
\end{thm}\index{Fibonacci!identità!Vajada}\index{Vajada}
\begin{proof}

	\proofpart{Prima dimostrazione}
	Riscriviamo il risultato dell'~\vref{eqn:FinBinet}
	\begin{align*}
		\Fib{n}={}                               & \dfrac{1}{a-b}\left(a^n-b^n\right)
		\intertext{da cui}
		\Fib{n+i}={}                             & \dfrac{1}{a-b}\left(a^{n+i}-b^{n+i}\right)     \\
		\Fib{n+j}={}                             & \dfrac{1}{a-b}\left(a^{n+j}-b^{n+j}\right)     \\
		\Fib{n+i+j}={}                           & \dfrac{1}{a-b}\left(a^{n+i+j}-b^{n+i+j}\right) \\
		\Fib{n+i}\Fib{n+j}-\Fib{n}\Fib{n+i+j}={} & \dfrac{a^{n}b^{n}(a^i-b^i)(a^j-b^j)}{(a-b)^2}  \\
		\intertext{Per~\vref{lem:FibpropPhi}}
		={}                                      & \dfrac{(-1)^{n}(a^i-b^i)(a^j-b^j)}{(a-b)^2}    \\
		={}                                      & (-1)^{n}\Fib{i}\Fib{j}                         \\
	\end{align*}
	cvd.

	\proofpart{Seconda dimostrazione}
	\begin{align*}
		\intertext{Per il~\vref{thm:FibProdSomma}}
		\Fib{m}\Fib{n}={}                        & \dfrac{\Luc{n+m}-(-1)^n\Luc{m-n}}{5}                                            \\
		\intertext{Quindi}
		\Fib{n+i}\Fib{n+j}={}                    & \dfrac{\Luc{n+i+n+j}-(-1)^{n+j}\Luc{n+i-n-j}}{5}                                \\
		={}                                      & \dfrac{\Luc{2n+i+j}-(-1)^{n+j}\Luc{i-j}}{5}                                     \\
		\Fib{n}\Fib{n+i+j}={}                    & \dfrac{\Luc{n+i+n+j}-(-1)^{n+i+j}\Luc{n-n-i-j}}{5}                              \\
		={}                                      & \dfrac{\Luc{2n+i+j}-(-1)^{n+i+j}\Luc{-i-j}}{5}                                  \\
		\Fib{n+i}\Fib{n+j}-\Fib{n}\Fib{n+i+j}={} & \dfrac{\Luc{2n+i+j}-(-1)^{n+j}\Luc{i-j}-\Luc{2n+i+j}+(-1)^{n+i+j}\Luc{-i-j}}{5} \\
		={}                                      & \dfrac{-(-1)^{n+j}\Luc{i-j}+(-1)^{n+i+j}\Luc{-i-j}}{5}                          \\
		\intertext{Per il~\vref{thm:FibLucNeg}}
		\Luc{-i-j}={}                            & (-1)^{i+j}\Luc{i+j}                                                             \\
		={}                                      & \dfrac{-(-1)^{n+j}\Luc{i-j}+(-1)^{n+i+j}(-1)^{i+j}\Luc{i+j}}{5}                 \\
		={}                                      & \dfrac{-(-1)^{n+j}\Luc{i-j}+(-1)^{n+2i+2j}\Luc{i+j}}{5}                         \\
		={}                                      & \dfrac{-(-1)^{n+j}\Luc{i-j}+(-1)^{n}\Luc{i+j}}{5}                               \\
		={}                                      & (-1)^{n}\dfrac{\Luc{i+j}-(-1)^{j}\Luc{i-j}}{5}                                  \\
		={}                                      & (-1)^{n}\Fib{i}\Fib{j}                                                          \\
	\end{align*}
	cvd.
\end{proof}
\begin{thm}[Identità di
		d'Ocagne]\label{thm:FibdOcagne}\index{Fibonacci!identità!D'Ocagne}\index{D'Ocagne}
	Se $\Fib{n}$ è la successione di Fibonacci allora
	\begin{equation}
		\Fib{m}\Fib{n+1}-\Fib{n}\Fib{m+1}=(-1)^n\Fib{m-n}
	\end{equation}\label{eqn:FibdOcagne}
\end{thm}
\begin{proof}
	\begin{align*}
		\intertext{Per il~\vref{thm:FibProdSomma}}
		\Fib{m}\Fib{n}={}   & \dfrac{\Luc{n+m}-(-1)^n\Luc{m-n}}{5}                                                       \\
		\intertext{Quindi}
		\Fib{m}\Fib{n+1}={} & \dfrac{\Luc{n+m+1}-(-1)^{n+1}\Luc{m-n-1}}{5}                                               \\
		\Fib{n}\Fib{m+1}={} & \dfrac{\Luc{n+m+1}-(-1)^{n}\Luc{m-n+1}}{5}                                                 \\
		\Fib{m}\Fib{n+1}-\Fib{n}\Fib{m+1}
		={}                 & \dfrac{1}{5}\left[\Luc{n+m+1}-(-1)^{n+1}\Luc{m-n-1}-\Luc{n+m+1}+(-1)^{n}\Luc{m-n+1}\right] \\
		={}                 & \dfrac{1}{5}\left[(-1)^{n}\Luc{m-n-1}+(-1)^{n}\Luc{m-n+1}\right]                           \\
		={}                 & \dfrac{(-1)^{n}}{5}\left[\Luc{m-n+1}+\Luc{m-n-1}\right]                                    \\
		\intertext{Per l'~\vref{eqn:FibConvertiinLucas}}
		={}                 & (-1)^{n}\Fib{m-n}                                                                          \\
	\end{align*}
	cvd.
\end{proof}
\begin{thm}[Identità di
		Honsberger]\label{thm:FibHonsberger}\index{Fibonacci!identità!Honsberger}\index{Honsberger}
	Se $\Fib{n}$ è la successione di Fibonacci allora
	\begin{equation}
		\Fib{k-1}\Fib{n}+\Fib{k}\Fib{n+1}=\Fib{n+k}
	\end{equation}\label{eqn:FibHonsberger}
\end{thm}
\begin{proof}
	\begin{align*}
		\intertext{Per il~\vref{thm:FibProdSomma}}
		\Fib{m}\Fib{n}={}   & \dfrac{\Luc{n+m}-(-1)^n\Luc{m-n}}{5}                                                     \\
		\intertext{Quindi}
		\Fib{k-1}\Fib{n}={} & \dfrac{\Luc{n+k-1}-(-1)^n\Luc{k-n-1}}{5}                                                 \\
		\Fib{k}\Fib{n+1}={} & \dfrac{\Luc{n+k+1}-(-1)^{n+1}\Luc{k-n-1}}{5}                                             \\
		\Fib{k-1}\Fib{n}-\Fib{k}\Fib{n+1}=
		                    & \dfrac{1}{5}\left[\Luc{n+k-1}-(-1)^n\Luc{k-n-1}+\Luc{n+k+1}-(-1)^{n+1}\Luc{k-n-1}\right] \\
		={}                 & \dfrac{1}{5}\left[\Luc{n+k-1}-(-1)^n\Luc{k-n-1}+\Luc{n+k+1}+(-1)^{n}\Luc{k-n-1}\right]   \\
		={}                 & \dfrac{1}{5}\left[\Luc{n+k-1}+\Luc{n+k+1}\right]                                         \\
		\intertext{Per l'~\vref{eqn:FibConvertiinLucas}}
		={}                 & \Fib{n+k}
	\end{align*}
	cvd.
\end{proof}
\begin{thm}[Identità di Gelin
		Cesàro]\label{thm:FibGelinCesaro}\index{Fibonacci!identità!Gelin
		Cesàro}\index{Gelin Cesàro}
	Se $\Fib{n}$ è la successione di Fibonacci allora
	\begin{equation}
		\Fib{n}^4-\Fib{n-2}\Fib{n-1}\Fib{n+1}\Fib{n+2}=1
	\end{equation}\label{eqn:FibGelinCesaro}
\end{thm}
\begin{proof}
	\begin{align*}
		\intertext{Per il~\vref{thm:FibProdSomma}}
		\Fib{m}\Fib{n}={}     & \dfrac{\Luc{n+m}-(-1)^n\Luc{m-n}}{5}                                                                                  \\
		\intertext{Quindi}
		\Fib{n-2}\Fib{n+2}={} & \dfrac{\Luc{n-2+n+2}-(-1)^{n-2}\Luc{n+2-n+2}}{5}                                                                      \\
		={}                   & \dfrac{\Luc{2n}-(-1)^{n-2}\Luc{4}}{5}                                                                                 \\
		\Fib{n-1}\Fib{n+1}={} & \dfrac{\Luc{n-1+n+1}-(-1)^{n-1}\Luc{n+1-n+1}}{5}                                                                      \\
		={}                   & \dfrac{\Luc{2n}-(-1)^{n-1}\Luc{2}}{5}                                                                                 \\
		\Fib{n-2}\Fib{n-1}\Fib{n+1}\Fib{n+2}
		={}                   & \dfrac{1}{25}\left[\Luc{2n}-(-1)^{n-1}\Luc{2}\right]\left[\Luc{2n}-(-1)^{n-2}\Luc{4}\right]                           \\
		={}                   & \dfrac{1}{25}\left[\Luc{2n}^2-(-1)^{n-2}\Luc{4}\Luc{2n}-(-1)^{n-1}\Luc{2}\Luc{2n}-(-1)^{n-1+n-2}\Luc{2}\Luc{4}\right] \\
		={}                   & \dfrac{1}{25}\left[\Luc{2n}^2-7(-1)^{n}\Luc{2n}+3(-1)^{n}\Luc{2n}-21\right]                                           \\
		={}                   & \dfrac{1}{25}\left[\Luc{2n}^2-4(-1)^{n}\Luc{2n}-21\right]                                                             \\
		\intertext{Per l'~\vref{eqn:FibQuartaSomma}}
		\Fib{n}^4={}          & \dfrac{\Luc{4n}-4(-1)^n\Luc{2n}+6}{25}                                                                                \\
		\intertext{Ricapitolando}
		={}                   & \dfrac{1}{25}\left[\Luc{4n}-4(-1)^n\Luc{2n}+6-\Luc{2n}^2+4(-1)^{n}\Luc{2n}+21\right]                                  \\
		={}                   & \dfrac{1}{25}\left[\Luc{4n}-\Luc{2n}^2+27\right]                                                                      \\
		\intertext{Ma}
		\Luc{2n}^2={}         & \Luc{4n}+2                                                                                                            \\
		={}                   & \dfrac{1}{25}\left[\Luc{4n}-\Luc{4n}-2+27\right]                                                                      \\
		={}                   & \dfrac{1}{25}25=1                                                                                                     \\
	\end{align*}
	cvd.
\end{proof}
\section{Quadrati}
\begin{thm}[Somma e differenza di
		quadrati]\label{thm:Fibsommsdifferenzaquadrati}
	Se $\Fib{n}$ è la successione di Fibonacci allora
	\begin{equation}
		\begin{cases}
			\Fib{n+r}^2-\Fib{n}^2={}\Fib{r}\Fib{2n+r}\quad\text{$r$ pari}    \\
			\Fib{n+r}^2+\Fib{n}^2={}\Fib{r}\Fib{2n+r}\quad\text{$r$ dispari} \\
		\end{cases}
	\end{equation}
\end{thm}
\begin{proof}
	\proofpart{$r$ pari}
	Riscriviamo il risultato dell'~\vref{eqn:FinBinet}
	\begin{align*}
		\Fib{n+r}^2-\Fib{n}^2-\Fib{r}\Fib{2n+r}
		={} & \dfrac{\left(a^{n+r}-b^{n+r}\right)^2}{(a-b)^2}
		-\dfrac{\left(a^{n}-b{n}\right)^2}{(a-b)^2}
		-\dfrac{(a^r-b^r)\left(a^{2n+r}-b^{2n+r}\right)^2}{(a-b)^2} \\
		={} & \dfrac{(a^n-b^n)^2\left(a^rb^r-1 \right)}{(a-b)^2}    \\
		\intertext{Per il~\vref{lem:FibpropPhi} se $r$ pari}
		={} & 0                                                     \\
	\end{align*}
	\proofpart{$r$ dispari}
	\begin{align*}
		\Fib{n+r}^2+\Fib{n}^2-\Fib{r}\Fib{2n+r}
		={} & \dfrac{\left(a^{n+r}-b^{n+r}\right)^2}{(a-b)^2}
		+\dfrac{\left(a^{n}-b{n}\right)^2}{(a-b)^2}
		-\dfrac{(a^r-b^r)\left(a^{2n+r}-b^{2n+r}\right)^2}{(a-b)^2} \\
		={} & \dfrac{(a^n-b^n)^2\left(a^rb^r+1 \right)}{(a-b)^2}    \\
		\intertext{Per il~\vref{lem:FibpropPhi} se $r$ dispari}
		={} & 0                                                     \\
	\end{align*}
	cvd.
\end{proof}
\begin{thm}[Somma di quadrati]\label{thm:Fibdispari}
	Se $\Fib{n}$ è la successione di Fibonacci allora
	\begin{equation}
		\Fib{n}^2+\Fib{n+1}^2=\Fib{2n+1}
	\end{equation}\label{eqn:FibDispari}
\end{thm}
\begin{proof}

	\proofpart{Prima dimostrazione}

	Riscriviamo il risultato dell'~\vref{eqn:FinBinet}
	\begin{align*}
		\Fib{n}={}               & \dfrac{1}{a-b}\left(a^n-b^n\right)
		\intertext{da cui}
		\Fib{n}^2={}             & \dfrac{a^{2n}-2a^nb^n+b^{2n}}{(a-b)^2}                     \\
		\Fib{n+1}^2={}           & \dfrac{a^{2(n+1)}-2a^{n+1}b^{n+1}+b^{2(n+1)}}{(a-b)^2}     \\
		\Fib{n}^2+\Fib{n+1}^2={} & \dfrac{a^{2n}(a^2+1)-2a^nb^n(ab+1)+b^{2n}(b^2+1)}{(a-b)^2}
		\intertext{Per~\vref{lem:FibpropPhi}}
		={}                      & \dfrac{a^{2n}(a-b)a+b^{2n}(b-a)b}{(a-b)^2}                 \\
		={}                      & \dfrac{1}{a-b}\left(a^{2n+1}-b^{2n+1}\right)               \\
		={}                      & \Fib{2n+1}
	\end{align*}
	cvd.

	\proofpart{Seconda dimostrazione}

	\begin{align*}
		\intertext{Per il~\vref{thm:FibCassini}}
		\Fib{n}^2={}             & \Fib{n-1}\Fib{n+1}-(-1)^n                        \\
		\intertext{Quindi}
		\Fib{n}^2+\Fib{n+1}^2={} & \Fib{n-1}\Fib{n+1}-(-1)^n+\Fib{n+1}^2            \\
		={}                      & \Fib{n+1}\left[\Fib{n-1}+\Fib{n+1}\right]-(-1)^n \\
		\intertext{ma per il~\vref{thm:LucasToFibFibToLuc}}
		\Luc{n}={}               & \Fib{n+1}+\Fib{n-1}
		\intertext{Quindi}
		={}                      & \Fib{n+1}\Luc{n}-(-1)^n                          \\
		\intertext{Per il~\vref{thm:FibProdSomma}}
		={}                      & \Fib{n+1+n}+(-1)^n\Fib{n+1-n}-(-1)^n             \\
		={}                      & \Fib{2n+1}+(-1)^n\Fib{1}-(-1)^n                  \\
		={}                      & \Fib{2n+1}                                       \\
	\end{align*}
\end{proof}
\begin{thm}[Differenza di quadrati]\label{thm:FibConsecutivi}
	Se $\Fib{n}$ è la successione di Fibonacci allora
	\begin{equation}
		\Fib{n+2}^2-\Fib{n+1}^2=\Fib{n}\Fib{n+3}
	\end{equation}\label{eqn:FibConsecutivi}
\end{thm}
\begin{proof}

	\proofpart{Prima dimostrazione}

	Riscriviamo il risultato dell'~\vref{eqn:FinBinet}
	\begin{align*}
		\Fib{n}={}                 & \dfrac{1}{a-b}\left(a^n-b^n\right)
		\intertext{da cui}
		\Fib{n+2}^2-\Fib{n+1}^2={} & \dfrac{a^{2(n+2)}-2(ab)^{n+2}+b^{2(n+2)}}{(a-b)^2}-\dfrac{a^{2(n+1)}-2(ab)^{n+1}+b^{2(n+1)}}{(a-b)^2} \\
		={}                        & \dfrac{a^{2(n+1)}(a^2-1)+2(ab)^{n+1}(1-ab)+b^{2(n+1)}(b^2-1)}{(a-b)^2}
		\intertext{Per~\vref{lem:FibpropPhi}}
		={}                        & \dfrac{a^{2(n+1)}a+2(ab)^{n+1}2+b^{2(n+1)}b}{(a-b)^2}                                                 \\
		={}                        & \dfrac{a^{2n+3}+4(ab)^{n+1}+b^{2n+3}b}{(a-b)^2}                                                       \\
		\Fib{n}\Fib{n+3}={}        & \dfrac{1}{a-b}\left(a^n-b^n\right)\dfrac{1}{a-b}\left(a^{n+3}-b^{n+3}\right)                          \\
		={}                        & \dfrac{(a^n-b^n)(a^{n+3}-b^{n+3})}{(a-b)^2}                                                           \\
		={}                        & \dfrac{(a^n-b^n)(a^{n}a^3-b^{n}b^3)}{(a-b)^2}
		\intertext{Per~\vref{lem:FibpropPhi}}
		={}                        & \dfrac{(a^n-b^n)[a^{n}(a^2+a)-b^{n}(b^2+b)]}{(a-b)^2}                                                 \\
		={}                        & \dfrac{a^{2n+1}(a+1)-(ab)^n(a^2+a+b^2+b)+b^{2n+1}(b+1)}{(a-b)^2}                                      \\
		={}                        & \dfrac{a^{2n+1}a^2-(ab)^n(a+b+2+a+b)+b^{2n+1}b^2}{(a-b)^2}                                            \\
		={}                        & \dfrac{a^{2n+3}+4ab(ab)^n+b^{2n+3}}{(a-b)^2}                                                          \\
		={}                        & \dfrac{a^{2n+3}+4(ab)^{n+1}+b^{2n+3}}{(a-b)^2}                                                        \\
	\end{align*}
	cvd.

	\proofpart{Seconda dimostrazione}
	\begin{align*}
		\intertext{per il~\vref{cor:FibpotSomma}}
		\Fib{n+2}^2={}      & \dfrac{\Luc{2n+4}-2(-1)^{n+2}}{5}                                                      \\
		={}                 & \dfrac{\Luc{2n+4}-2(-1)^{n}}{5}                                                        \\
		\Fib{n+1}^2={}      & \dfrac{\Luc{2n+2}-2(-1)^{n+1}}{5}                                                      \\
		={}                 & \dfrac{\Luc{2n+2}-2(-1)^{n}}{5}                                                        \\
		\Fib{n+2}^2-\Fib{n+1}^2
		={}                 & \dfrac{\Luc{2n+4}-2(-1)^{n}-\Luc{2n+2}-2(-1)^{n}}{5}                                   \\
		={}                 & \dfrac{\Luc{2n+4}-\Luc{2n+2}-4(-1)^{n}}{5}                                             \\
		\intertext{Per il~\vref{thm:LucFibSommaprodotto}}
		\Luc{2n+4}={}       & \dfrac{5\Fib{4}\Fib{2n}+\Luc{4}\Luc{2n}}{2}                                            \\
		={}                 & \dfrac{15\Fib{2n}+7\Luc{2n}}{2}                                                        \\
		\Luc{2n+2}={}       & \dfrac{5\Fib{2}\Fib{2n}+\Luc{2}\Luc{2n}}{2}                                            \\
		={}                 & \dfrac{5\Fib{2n}+3\Luc{2n}}{2}                                                         \\
		\intertext{Quindi}
		={}                 & \dfrac{1}{5}\left[\dfrac{15\Fib{2n}+7\Luc{2n}-5\Fib{2n}-3\Luc{2n}}{2}-4(-1)^{n}\right] \\
		={}                 & \dfrac{1}{5}\left[5\Fib{2n}+2\Luc{2n}-4(-1)^{n}\right]                                 \\
		\intertext{Per il~\vref{thm:LucFibSommaprodotto}}
		\Fib{n}\Fib{n+3}={} & \Fib{n}\dfrac{2\Luc{n}+4\Fib{n}}{2}                                                    \\
		={}                 & \dfrac{2\Fib{n}\Luc{n}+4\Fib{n}^2}{2}                                                  \\
		={}                 & \Fib{n}\Luc{n}+2\Fib{n}^2
		\intertext{per il~\vref{cor:FibpotSomma}}
		2\Fib{n}^2={}       & \dfrac{2\Luc{2n}-4(-1)^{n}}{5}                                                         \\
		\intertext{per il~\vref{thm:FibProdSomma}}
		\Fib{n}\Luc{n}={}   & \Fib{n+n}+(-1)^n\Fib{n-n}                                                              \\
		\Fib{n}\Luc{n}={}   & \Fib{2}                                                                                \\
		\intertext{Quindi}
		\Fib{n}\Fib{n+3}={} & \dfrac{1}{5}\left[5\Fib{2n}+2\Luc{2n}-4(-1)^{n}\right]                                 \\
	\end{align*}
	cvd.
\end{proof}

\begin{thm}[Somma di quadrati]
	Se $\Fib{n}$ è la successione di Fibonacci allora:
	\begin{equation}
		\Fib{n}^2+\Fib{n+1}^2=\Fib{n}\Fib{n+2}+\Fib{n-1}\Fib{n+1}
	\end{equation}\label{eqn:FibquadratiConsecutivi}
\end{thm}
\begin{proof}

	\proofpart{Prima dimostrazione}

	Riscriviamo il risultato dell'~\vref{eqn:FinBinet}
	\begin{align*}
		\Fib{n}={}                             & \dfrac{1}{a-b}\left(a^n-b^n\right)
		\intertext{da cui}
		\Fib{n}^2={}                           & \dfrac{a^{2n}-2(ab)^n+b^{2n}}{(a-b)^2}                     \\
		\Fib{n+1}^2={}                         & \dfrac{a^{2(n+1)}-2(ab)^{(n+1)}+b^{2(n+1)}}{(a-b)^2}       \\
		\Fib{n}^2+\Fib{n+1}^2={}               & \dfrac{a^{2n}(a^2+1)-2(ab)^n(ab+1)+b^{2n}(b^2+1)}{(a-b)^2} \\
		\intertext{Per~\vref{lem:FibpropPhi}}
		={}                                    & \dfrac{a^{2n}(a^2+1)+b^{2n}(b^2+1)}{(a-b)^2}               \\
		\Fib{n}\Fib{n+1}={}                    & \dfrac{(a^{n}-b^{n})((a^{n+2}-b^{n+2})}{(a-b)^2}           \\
		\Fib{n-1}\Fib{n+1}={}                  & \dfrac{(a^{n+1}-b^{n+1})((ba^{n}-ab^{n})}{(a-b)^2}         \\
		\Fib{n}\Fib{n+2}+\Fib{n-1}\Fib{n+1}={} &
		\dfrac{ba^{2n+1}(a^2+1)-(ab)^n(a^2+b^2)(ab+1)+ab^{2n+1}(b^2+1)}{(a-b)^2}                            \\
		\intertext{Per~\vref{lem:FibpropPhi}}
		={}                                    & \dfrac{a^{2n}(a^2+1)+b^{2n}(b^2+1)}{(a-b)^2}               \\
	\end{align*}
	Da cui la tesi.

	\proofpart{Seconda dimostrazione}
	\begin{align*}
		\intertext{Per il~\vref{cor:FibpotSomma}}
		\Fib{n}^2+\Fib{n+1}^2
		={}                   & \dfrac{\Luc{2n}-2(-1)^n}{5}+\dfrac{\Luc{2n+2}-2(-1)^{n+1}}{5} \\
		={}                   & \dfrac{\Luc{2n}+\Luc{2n+2}}{5}                                \\
		\intertext{Per il~\vref{thm:FibProdSomma}}
		\Fib{n}\Fib{n+2}+
		\Fib{n+1}\Fib{n-1}={} & \dfrac{\Luc{n+n+2}-(-1)^n\Luc{n+2-n}}{5}
		\dfrac{\Luc{n+1+n-1}-(-1)^{n-1}\Luc{n+1-n+1}}{5}                                      \\
		\Fib{n+1}\Fib{n-1}={} & \dfrac{\Luc{2n+2}+3(-1)^n}{5}
		\dfrac{\Luc{2n}-3(-1)^{n}}{5}                                                         \\
		={}                   & \dfrac{\Luc{2n+2}+\Luc{2n}}{5}                                \\
	\end{align*}
	cvd.
\end{proof}
\section{Proprietà}
\begin{thm}[Tre numeri consecutivi]
	Tre numeri di Fibonacci consecutivi non possono essere i lati di un
	triangolo.
\end{thm}\index{Fibonacci!triangolo}
\begin{proof}
	Supponiamo che $\Fib{n}$, $\Fib{n+1}$ e $\Fib{n+2}$ siano i lati di un
	triangolo. Allora per la formula di Erone avremo, indicando con $p$
	il semiperimetro e $a$, $b$ e $c$ i lati:
	\begin{align*}
		S^2={} & p(p-a)(p-b)(p-c)                       \\
		\intertext{Ma}
		p={}   & \dfrac{\Fib{n}+\Fib{n+1}+\Fib{n+2}}{2} \\
		p={}   & \dfrac{\Fib{n+2}+\Fib{n+2}}{2}         \\
		p={}   & \Fib{n+2}                              \\
		p-c={} & \Fib{n+2}-\Fib{n+2}=0                  \\
	\end{align*}
	Quindi il triangolo ha area zero, assurdo.
\end{proof}
\begin{thm}[Somme di numeri di Fibonacci]
	Se $\Fib{n}$ è la successione di Fibonacci allora:
	\begin{equation}
		\sum_{k=1}^{n}\Fib{k}=\Fib{n+2}-1
	\end{equation}\label{eqn:FibSommaNumeri}
\end{thm}\index{Fibonacci!somma}
\begin{proof}
	\begin{align*}
		\intertext{Per induzione su $n$}
		\Fib{1}={}                         & \Fib{1+2}-1=1       \\
		\intertext{Supponiamola vera per $n-1$ proviamola per $n$}
		\sum_{k=1}^{n-1}\Fib{k}+\Fib{n}={} & \Fib{n+1}+\Fib{n}-1 \\
		\sum_{k=1}^{n}\Fib{k}={}           & \Fib{n+2}-1         \\
	\end{align*}
	cvd.
\end{proof}
\begin{thm}[$\Fib{3n}$ è pari]
	Se $\Fib{n}$ è la successione di Fibonacci allora:
	$\Fib{3n}$ è pari qualunque $n\geq1$\label{eqn:Fib3npari}
\end{thm}\index{Fibonacci!Valori pari}
\begin{proof}
	Per induzione su $n$. Per $n=1$ $\Fib{3}=2$ quindi è vero. Supponiamolo
	vero per $n$ proviamolo per $n+1$
	\begin{align*}
		\Fib{3(n+1)}={} & \Fib{3n+3}                     \\
		={}             & \Fib{3n+2}+\Fib{3n+1}          \\
		={}             & \Fib{3n+1}+\Fib{3n}+\Fib{3n+1} \\
		={}             & 2\Fib{3n+1}+\Fib{3n}           \\
	\end{align*}
	Dato che $\Fib{3n}$ è pari per ipotesi induttiva $\Fib{3n+3}$ è pari.
\end{proof}
\begin{thm}[$\Fib{4n}$ è multiplo di tre]
	Se $\Fib{n}$ è la successione di Fibonacci allora:
	$\Fib{4n}$ è multiplo $3$\label{eqn:Fib4multiploditre}
\end{thm}\index{Fibonacci!Valori multipli di tre}
\begin{proof}
	Per induzione su $n$. Per $n=1$ $\Fib{4}=3$ quindi è vero. Supponiamolo
	vero per $n$ proviamolo per $n+1$
	\begin{align*}
		\Fib{4(n+1)}={} & \Fib{4n+4}                        \\
		={}             & \Fib{4n+3}+\Fib{4n+2}             \\
		={}             & \Fib{4n+2}+\Fib{4n+1}+\Fib{4n+2}  \\
		={}             & 2\Fib{4n+2}+\Fib{4n+1}            \\
		={}             & 2(\Fib{4n+1}+\Fib{4n})+\Fib{4n+1} \\
		={}             & 3\Fib{4n+1}+2\Fib{4n}             \\
	\end{align*}
	Dato che $\Fib{4n}$ è multiplo di tre per ipotesi induttiva $\Fib{4n+4}$ è
	multiplo di tre.
\end{proof}
\begin{thm}[$\Fib{6n\pm 1}$ è della forma $4k+1$]
	Se $\Fib{n}$ è la successione di Fibonacci allora:
	$\Fib{6n+1}$ e $\Fib{6n-1}$ sono della forma $4k+1$
\end{thm}\index{Fibonacci!Valori multipli}
\begin{thm}[Due termini successivi sono coprimi]\label{thm:FibonacciCoPrimi}
	Se $\Fib{n}$ è la successione di Fibonacci allora:
	$\Fib{n}$ e $\Fib{n+1}$ sono coprimi.
\end{thm}\index{Fibonacci!Coprimi}\index{Coprimi}\cite{Gregorio2022}
\begin{proof}
	Abbiamo
	\begin{align*}
		\Fib{0}={}   & 0                 \\
		\Fib{1}={}   & 1                 \\
		\Fib{n+2}={} & \Fib{n+1}+\Fib{n} \\
		\intertext{Poniamo che $\Fib{n+1}$ e $\Fib{n}$ siano coprimi e che $n$ sia
			il valore minimo. Abbiamo che}
		n>           & 1                 \\
		\intertext{Perché $\Fib{0}$ e $\Fib{1}$ sono coprimi fra di loro come anche
			$\Fib{1}$ e $\Fib{2}$ }
		\intertext{Per ipotesi quindi}
		\Fib{n+1}={} & ad                \\
		\Fib{n}      & =bd
		\intertext{con $a,b,d$ interi e $d>1$}
		\Fib{n-1}={} & \Fib{n+1}-\Fib{n} \\
		={}          & ad-bd             \\
		={}          & (a-b)d            \\
		\intertext{Quindi $\Fib{n-1}$ e $\Fib{n}$ non sono coprimi m $n$ era il valore
			minimo per cui $\Fib{n+1}$ e $\Fib{n}$ erano coprimi.}
	\end{align*}
	Cvd.
\end{proof}

% !TeX root = appuntifibo.tex
% !BIB TS-program = biber
% !TeX encoding = UTF-8
% !TeX spellcheck = it_IT
% !TeX program = lualatex
\chapter{Fibonacci valori iniziali qualunque}
\section{Definizione}
\begin{defn}[Fibonacci, valori Iniziali 
qualunque]\index{Fibonacci!iniziali qualunque!definizione}
	\begin{align}
		\Gib{0}={}&h\notag\\
		\Gib{1}={}&k\notag\\
		\Gib{n}={}&\Gib{n-1}+\Gib{n-2}\quad n>2\label{eqn:FiboGendef}
	\end{align}
\end{defn}
\section{Formula di Binet per valori iniziali qualunque}
\begin{thm}[Formula di Binet 
per valori iniziali 
qualunque]\label{thm:FormulaBinetgeneralizzata}\index{Formula!Binet!per valori 
iniziali qualunque}
Se $\Gib{0}={}h$ e $\Gib{1}={}k$ allora \begin{equation}
	\Gib{n}={}\dfrac{k-bh}{a-b}a^n+\dfrac{ah-k}{a-b}b^n
\end{equation}\label{eqn:FormulaBinetgeneralizzata}
\end{thm}
\begin{proof}
Poniamo che \begin{equation}
	\Gib{n}=xa^n+yb^n
\end{equation} Se
\begin{equation*}
	\left\{
	\begin{array}{l}
		a^0x+b^0y=h\\ a^1x+b^1y=k
	\end{array}
	\right.
\end{equation*}
che risolto porta
\begin{equation*}
	\left\{
	\begin{array}{l}
		x=\dfrac{k-bh}{a-b}\\y=\dfrac{ah-k}{a-b}\\
	\end{array}
	\right.
\end{equation*}
Riassumendo
 \begin{equation*}
	\Gib{n}={}\dfrac{k-bh}{a-b}a^n+\dfrac{ah-k}{a-b}b^n
\end{equation*}
Da cui la tesi.
\end{proof}
\begin{commento}
	Se $\Gib{0}={}2h$ e $\Gib{1}={}2k$ allora \begin{equation}
		\Gib{n}={}2\dfrac{k-bh}{a-b}a^n+2\dfrac{k-ah}{a-b}b^n
	\end{equation}
La somma di due pari è sempre pari quindi la successione è $PPPPPP$
\end{commento}
\begin{commento}
	Se $\Gib{0}={}2h$ e $\Gib{1}={}2k+1$ allora \begin{equation}
		\Gib{n}={}\dfrac{1+2k-2bh}{a-b}a^n+\dfrac{2ah-2k-1}{a-b }b^n
	\end{equation}
La somma è fra un numero pari e un dispari è dispari. Segue la somma tra due dispari che è pari e quello che segue è un dispari. La successione che otteniamo  è $PDDPDDP$
\end{commento}
\begin{commento}
	Se $\Gib{1}={}2h+1$ e $\Gib{2}={}2k$ allora \begin{equation}
		\Gib{n}={}\dfrac{2k-b(2h+1)}{a-b}a^n+\dfrac{a(2h+1)-2k}{a-b}b^n
	\end{equation}
La somma è tra un numero dispari e un pari. In questo caso otteniamo $DPDDPDDP$
\end{commento}
\begin{commento}
	Se $\Gib{0}={}2h+1$ e $\Gib{1}={}2k+1$ allora \begin{equation}
		\Gib{n}={}\dfrac{2k-b(2h+1)}{a-b}a^n-\dfrac{a(2h+1)-2k-1}{a-b}b^n
	\end{equation}
\end{commento}
La tabella \vref{tab:seqnumpar} riassume quanto detto.
\begin{commento}
	Se $\Gib{0}={}h$ e $\Gib{1}={}h$ allora \begin{equation}
		\Gib{n}={}h\dfrac{a}{a-b}a^n-h\dfrac{b}{a-b}b^n
	\end{equation}
\end{commento}
\begin{commento}
	Se $\Gib{0}={}2h$ e $\Gib{1}={}h$ allora \begin{equation}
		\Gib{n}={}ha^n+hb^n
	\end{equation}
\end{commento}
\begin{table}
	\centering
	\begin{tabular}{ccc}
\toprule
$h$&$k$&\\
\midrule
	$P$& $P$ &$PPPPPPPPP$ \\
	$P$& $D$ &$PDDPPDDPD$  \\
	$D$& $P$ &$DPDDPDDPD$  \\
	$D$& $D$ &$DDPDDPDDP$  \\
\bottomrule
\end{tabular}
	\caption{Sequenze pari e dispari}
	\label{tab:seqnumpar}
\end{table}
\section{Proprietà}
Possiamo con qualche calcolo, verificare che
\begin{align*}
\Gib{0}={}&h\\
\Gib{1}={}&k\\
\Gib{2}={}&k+h=\Fib{2}k+\Fib{1}h\\
\Gib{3}={}&2k+h=\Fib{3}k+\Fib{2}h\\
\Gib{4}={}&3k+2h=\Fib{4}k+\Fib{3}h\\
\Gib{5}={}&5k+3h=\Fib{5}k+\Fib{4}h                                              
\end{align*} 
\begin{thm}[Derivazione]
	Se $\Gib{n}$ è una successione di Fibonacci per valori iniziali qualunque  
	con $\Gib{0}=h$ 
	e $\Gib{1}=k $ allora
	\begin{equation}
		\Gib{n}=k\Fib{n}+h\Fib{n-1}
	\end{equation}\label{thm:FibGenDer}
	Dove $\Fib{n}$ è la successione di Fibonacci
\end{thm}
\begin{proof}
Per induzione
\begin{align*}
\intertext{proviamola per n+1}
		\Gib{n+1}={}&\Gib{n}+\Gib{n-1}\\
\intertext{per ipotesi induttiva}
\Gib{n}={}&\Fib{n}k+\Fib{n-1}h\\
\Gib{n}={}&\Fib{n-1}k+\Fib{n-2}h\\
\Gib{n}+\Gib{n-1}={}&\Fib{n}k+\Fib{n-1}h+\Fib{n-1}k+\Fib{n-2}h\\
={}&(\Fib{n}+\Fib{n-1})k+(\Fib{n-1}+\Fib{n-2})h\\
={}&\Fib{n+1}k+\Fib{n}h\\
\end{align*}
cvd.
\end{proof}
\begin{thm}[Limite successione]
	Se $\Gib{n}$ è la successione di Fibonacci per valori iniziali qualunque 
	allora 
	\begin{equation}
		\lim_{n\to\infty}\dfrac{\Gib{n+1}}{\Gib{n}}=\varphi
	\end{equation}\label{eqn:FibLimRapGen}
\end{thm}
\begin{proof}
	\begin{align*}
		\lim_{n\to\infty}\dfrac{\Gib{n+1}}{\Gib{n}}={}&ab\lim_{n\to\infty}\dfrac{a^n(bh-k)+b^n(k-ah)}{ba^n(bh-k)+ab^n(k-ah)}\\
		={}&ab\lim_{n\to\infty}\dfrac{a^n[(bh-k)+\dfrac{b^n}{a^n}(k-ah)]}{a^n[b(bh-k)+a\dfrac{b^n}{a^n}(k-ah)]}\\
		\intertext{ma}
	\lim_{n\to\infty}\left(\dfrac{b}{a}\right)^n={}&0\\
={}&ab\lim_{n\to\infty}\dfrac{bh-k}{b(bh-k)}\\
={}&ab\dfrac{1}{b}=a=\varphi\\
	\end{align*}
 \end{proof}
\begin{thm}[Limite rapporto]
	Se $\Fib{n}$ è la successione di Fibonacci e $\Gib{n}$ è quella per valori 
	iniziali qualunque allora 
	\begin{equation}
		\lim_{n\to\infty}\dfrac{\Gib{n}}{\Fib{n}}=k-bh
	\end{equation}\label{eqn:FibLimFibGib}
\end{thm}~\cite{Horadam_1961}
\begin{proof}
	\begin{align*}
		\lim_{n\to\infty}\dfrac{\Gib{n}}{\Fib{n}}={}&\lim_{n\to\infty}\dfrac{a^n(k-bh)+b^n(ah-k)}{a^n-b^n}\\
		={}&\lim_{n\to\infty}\dfrac{a^n\left[k-bh+\left(\dfrac{b}{a}\right)^n(ak-k)\right]}{a^n\left[1-\left(\dfrac{b}{a}\right)^n\right]}\\
		={}&k-bh\\
		\intertext{Dato che}
		\lim_{n\to\infty}\left(\dfrac{b}{a}\right)^{n}={}&0\\
	\end{align*}
	Come si voleva dimostrare.
\end{proof}
\begin{thm}[Limite rapporto]
	Se  $\Gib{n}$ è la successione per valori iniziali qualunque allora 
	\begin{equation}
		\lim_{n\to\infty}\dfrac{\Gib{n}}{\Gib{n-i}}=a^i
	\end{equation}\label{eqn:GibLimdif}
\end{thm}~\cite{Horadam_1961}
\begin{proof}
	\begin{align*}
		\lim_{n\to\infty}\dfrac{\Gib{n}}{\Gib{n-i}}={}&	\lim_{n\to\infty}-\dfrac{a^ib^i[a^n(bh-k)+b^n(k-ah)]}{a^ib^n(ah-k)+a^nb^i(k-bh)}\\
		={}&\lim_{n\to\infty}-\dfrac{a^n\left\{a^ib^i\left[bh-k\right]\left(\dfrac{b}{a}\right)^n(k-ah)\right\}}{a^n\left\{a^i\left(\dfrac{b}{a}\right)^n(bh-k)+b^i(k-ah)\right\}}\\
		={}&\dfrac{a^ib^i(k-bh)}{b^i(k-bh)}\\
		={}&a^i
		\intertext{Dato che}
		\lim_{n\to\infty}\left(\dfrac{b}{a}\right)^{n}={}&0\\
	\end{align*}
\end{proof}
\begin{lem}\label{lem:PropPhiGen}
	\begin{equation}
		(k-bh)(k-ak)=k^2-hk-h^2
	\end{equation}
\end{lem}
\begin{proof}
	\begin{align*}
			(k-bh)(k-ak)=&abh^2-(a+b)hk+k^2
		\intertext{per~\vref{lem:FibpropPhi}}
		=&k^2-hk-h^2
	\end{align*}
cvd.
\end{proof}
\begin{thm}[Identità di Cassini 
per valori iniziali qualunque]\label{thm:identitàCassiniper valori iniziali 
qualunque}
	Se $\Gib{n}$ è la successione di Fibonacci per valori iniziali qualunque 
	allora 
	\begin{equation}
		\Gib{n-1}\cdot\Gib{n+1}-\Gib{n}^2=(-1)^n(k^2-hk-h^2)
	\end{equation}\label{eqn:FibQuadratoGen}
\end{thm}\index{Fibonacci!identità!Cassini per valori iniziali qualunque}
\begin{proof}
\begin{align*}
\Gib{n}=&{}\dfrac{k-bh}{a(a-b)}a^n+\dfrac{ah-k}{b(a-b)}b^n\\
\Gib{n-1}=&{}\dfrac{k-bh}{a(a-b)}a^{n-1}+\dfrac{ah-k}{b(a-b)}b^{n-1}\\
\Gib{n+1}=&{}\dfrac{k-bh}{a(a-b)}a^{n+1}+\dfrac{ah-k}{b(a-b)}b^{n+1}\\
\Gib{n-1}\cdot\Gib{n+1}-\Gib{n}^2=&a^{n-1}b^{n-1}(k-bh)(ah-k)\\
	\intertext{per~\vref{lem:FibpropPhi}}
	=&(-1)^{n}(k-bh)(k-ah)
	\intertext{per~\vref{lem:PropPhiGen}}
	=&(-1)^n(k^2-hk-h^2)
\end{align*}
cvd.
\end{proof}
\begin{thm}[Identità di Catalan per valori iniziali 
qualunque]\label{thm:fibCatalanGen}
	Se $\Gib{n}$ è la successione di Fibonacci per valori iniziali qualunque 
	allora 
	\begin{equation}
		\Gib{n-r}\cdot\Gib{n+r}-\Gib{n}^2=(-1)^{n-r+1}(k^2-hk-h^2)\Gib{r}^2
	\end{equation}\label{eqn:fibCatalanGen}
\end{thm}\index{Fibonacci!identità!Catalan per valori iniziali qualunque}
\begin{proof}
	\begin{align*}
		\Gib{n}=&{}\dfrac{k-bh}{a(a-b)}a^n+\dfrac{ah-k}{b(a-b)}b^n\\
		\Gib{n-r}=&{}\dfrac{k-bh}{a(a-b)}a^{n-r}+\dfrac{ah-k}{b(a-b)}b^{n-r}\\
		\Gib{n+r}=&{}\dfrac{k-bh}{a(a-b)}a^{n+r}+\dfrac{ah-k}{b(a-b)}b^{n+r}\\
		\Gib{n-r}\cdot\Gib{n+r}-\Gib{n}^2=&a^{n-r}b^{n+r}(k-bh)(ah-k)\dfrac{a^{2r}-2a^rb^r+b^{2r}}{(a-b)^2}\\
		=&a^{n-r}b^{n+r}(k-bh)(ah-k)\Gib{r}^2\\
		\intertext{per~\vref{lem:FibpropPhi}}
		=&(-1)^{n-r+1}(k-bh)(k-ah)\Gib{r}^2\\
		\intertext{per~\vref{lem:PropPhiGen}}
		=&(-1)^{n-r+1}(k^2-hk-h^2)\Gib{r}^2
	\end{align*}
	cvd.
\end{proof}
\begin{thm}[Identità di Vajada per valori iniziali 
qualunque]\label{thm:fibVajdaGen}
	Se $\Gib{n}$ è la successione di Fibonacci per valori iniziali qualunque 
	allora 
	\begin{equation}
		\Gib{n+i}\cdot\Gib{n+j}-\Gib{n}\cdot\Gib{n+i+j}=(-1)^{n}(k^2-hk-h^2)\Gib{i}\cdot\Gib{j}
	\end{equation}\label{eqn:fibVajdaGen}
\end{thm}\index{Fibonacci!identità!Vajada per valori iniziali qualunque}
\begin{proof}
	\begin{align*}
		\Gib{n}=&{}\dfrac{k-bh}{a(a-b)}a^n+\dfrac{ah-k}{b(a-b)}b^n\\
		\Gib{n+i}=&{}\dfrac{k-bh}{a(a-b)}a^{n+i}+\dfrac{ah-k}{b(a-b)}b^{n+i}\\
		\Gib{n+j}=&{}\dfrac{k-bh}{a(a-b)}a^{n+j}+\dfrac{ah-k}{b(a-b)}b^{n+j}\\
		\Gib{n+i+j}=&{}\dfrac{k-bh}{a(a-b)}a^{n+i+j}+\dfrac{ah-k}{b(a-b)}b^{n+i+j}\\
		\Gib{n+i}\cdot\Gib{n+j}-\Gib{n}\cdot\Gib{n+i+j}=&\dfrac{a^nb^n(a^i-b^i)(a^j-b^j)(k-ah)(k-bh)}{(a-b)^2}\\
		\intertext{per~\vref{lem:FibpropPhi}}
		=&(-1)^{n}(k-bh)(k-ah)\Gib{i}\cdot\Gib{j}\\
		\intertext{per~\vref{lem:PropPhiGen}}
		=&(-1)^{n}(k^2-hk-h^2)\Gib{i}\cdot\Gib{j}\\
	\end{align*}
	cvd.
\end{proof}
\section{Proprietà esclusive successione di Fibonacci}
\begin{thm}[Sistema caratteristico]\label{thm:FibSistCaratteristico}
Se $\Gib{n}$ è una generica successione di Fibonacci con
\[\begin{cases}
\Gib{0}=h\\
\Gib{1}=k\\
\end{cases}\] il sistema 
\begin{equation*}
	\left\{
\begin{array}{l}
	(a^2+1)(k-bh)(k-bh-1)=0\\
	(b^2+1)(ah-k)(ah-k+1)=0
\end{array}
\right.
\end{equation*}
ha come soluzione intera solo \[\begin{cases}
	h=0\\
	k=1\\
\end{cases}\]
\end{thm}
\begin{proof}
Questo sistema ammette quattro soluzioni che sono
\begin{enumerate}
	\item $\left\{
	\begin{array}{l}
		h=0\\
		k=0
	\end{array}
	\right.$
	\item $\left\{
	\begin{array}{l}
		h=0\\
		k=1
	\end{array}
	\right.$
	\item $\left\{
	\begin{array}{l}
		h=\dfrac{1}{a-b}\\
		k=\dfrac{a}{a-b}
	\end{array}
	\right.$
	\item $\left\{
	\begin{array}{l}
		h=-\dfrac{1}{a-b}\\
		k=-\dfrac{b}{a-b}
	\end{array}
	\right.$
\end{enumerate}
Discutiamo le soluzioni. La prima non è accettabile. La seconda sono i valori iniziali della successione di Fibonacci.
La terza e la quarta non sono accettabili perché non intere. 
\end{proof}
\begin{thm}[Dispari]\label{thm:FibdispariGen}
	Se $\Gib{n}$ è la successione di Fibonacci per valori iniziali qualunque e 
	vale 
	\begin{equation}
		\Gib{n}^2+\Gib{n+1}^2-\Gib{2n+1}=0 
	\end{equation}\label{eqn:FibDispariGen} allora 
\[\begin{cases}
	h=0\\
	k=1\\
\end{cases}\]
\end{thm}
\begin{proof}
\begin{align*}
\Gib{n}^2+\Gib{n+1}^2-\Gib{2n+1}={}&a^{2n}\dfrac{(a^2+1)(bh-k)(bh-k+1)}{\left(a-b\right)^2}\\
+&b^{2n}\dfrac{(b^2+1)(ah-k)(ah-k+1)}{\left(a-b\right)^2}\\
\intertext{l'uguaglianza precedente vale zero se $h$ e $k$ sono soluzioni del sistema seguente }
&	\left\{
\begin{array}{l}
	(a^2+1)(k-bh)(k-bh-1)=0\\
	(b^2+1)(ah-k)(ah-k+1)=0
\end{array}
\right.\\
\end{align*}
Quindi per~\vref{thm:FibSistCaratteristico} è la successione di Fibonacci ordinaria.
\end{proof}
Utilizzando il~\vref{thm:FibdispariGen} e il~\vref{thm:Fibdispari} otteniamo
\begin{thm}[Unicità]
	\begin{equation}
		\Gib{n}^2+\Gib{n+1}^2=\Gib{2n+1} 
	\end{equation} se e solo la successione è quella ordinaria di Fibonacci. 
\end{thm}
\begin{thm}[Proprietà]
	Se $\Gib{n}$ è una  successione di Fibonacci per valori iniziali qualunque
	\begin{equation}
		\Gib{n+2}-2\Gib{n}-\Gib{n-1}=0 
	\end{equation} 
\begin{equation}
	\Gib{n+1}-2\Gib{n}-\Gib{n-2}=0 
\end{equation} 
\end{thm}~\cite{Horadam_1961}
\begin{proof}
	\begin{align*}
		\Gib{n+2}-2\Gib{n}-\Gib{n-1}=&\\
		=&\dfrac{a^{n-1}(a^3-2a-1)(k-bh)}{a-b}+\dfrac{b^{n-1}(b^3-2b-1)(ah-k)}{a-b}
		\intertext{per~\vref{lem:FibpropPhi}}
		a^3-2a-1=&\\
		=&aa^2-2a-1	\\
		=&a(a+1)-2a-1\\
		=&a^2+a-2a-1\\
		=&a^2-a-1\\
		=&0
		\intertext{Analogamente}
		b^3-2b-1=&0\\
	\end{align*}
Quindi la tesi. Come prima si dimostra che
	\begin{align*}
	\Gib{n+2}-2\Gib{n}-\Gib{n-1}=&\\
	=&\dfrac{a^{n-2}(a^3-2a-1)(k-bh)}{a-b}+\dfrac{b^{n-2}(b^3-2b-1)(ah-k)}{a-b}
\end{align*}
e procedendo in maniera analoga otteniamo la tesi.

Cvd.
\end{proof}

% !TeX root = appuntifibo.tex
% !BIB TS-program = biber
% !TeX encoding = UTF-8
% !TeX spellcheck = it_IT
\chapter{Rettangoli di Fibonacci}
\section{Definizione}
\begin{defn}[Rettangoli di Fibonacci]\index{Fibonacci!Rettangoli!definizione}
ccdg
	Se $\Fib{n}$ è la successione di Fibonacci allora avremo:
	\begin{equation}
		\Ret{n}=\Fib{n}\Fib{n+1}
	\end{equation}	
\end{defn}\cite{A001654}
% !TeX root = appuntifibo.tex
% !BIB TS-program = biber
% !TeX encoding = UTF-8
% !TeX spellcheck = it_IT
% !TeX program = lualatex
\chapter{Matrici}
\section{Successioni di Fibonacci }
\begin{thm}[Forma vettoriale]\index{Fibonacci!vettore}
	Se $\Fib{n}$ è la successione di Fibonacci allora avremo:
	\begin{equation}
		\begin{bmatrix}
			\Fib{n}\\\Fib{n+1}
		\end{bmatrix}=\begin{bmatrix}
			0&1\\ 1&1\\
		\end{bmatrix}^n\begin{bmatrix}
		\Fib{0}\\\Fib{1}
	\end{bmatrix}
	\end{equation}
\end{thm}
\begin{thm}[Forma matriciale]\index{Fibonacci!matrice}
Se $\Fib{n}$ è la successione di Fibonacci allora avremo:
\begin{equation}
Q^n=\begin{bmatrix}
	\Fib{n+1}&\Fib{n}\\\Fib{n}&\Fib{n-1}
\end{bmatrix}=\begin{bmatrix}
	1&1\\ 1&0\\
\end{bmatrix}^n
\end{equation}
\end{thm}\cite{Gould1981}
\begin{proof}
	Dimostriamola per induzione su  $n$.
	
	Per $n=1$
	\[Q^1=\begin{bmatrix}
		\Fib{2}&\Fib{1}\\\Fib{1}&\Fib{0}
	\end{bmatrix}=\begin{bmatrix}
		1&1\\ 1&0\\
	\end{bmatrix}^1\\
	\]
	
	Supponiamola vera per $n=k$ proviamola per $n=k+1$
	\begin{align*}
		Q^k=\begin{bmatrix}
		\Fib{k+1}&\Fib{k}\\\Fib{k}&\Fib{k-1}
	\end{bmatrix}=&\begin{bmatrix}
		1&1\\ 1&0\\
	\end{bmatrix}^k\\
\intertext{Moltiplicando}
	\begin{bmatrix}
	\Fib{k+1}&\Fib{k}\\\Fib{k}&\Fib{k-1}
\end{bmatrix}\begin{bmatrix}
1&1\\ 1&0\\
\end{bmatrix}=&\begin{bmatrix}
	1&1\\ 1&0\\
\end{bmatrix}^k\begin{bmatrix}
1&1\\ 1&0\\
\end{bmatrix}\\
	\begin{bmatrix}
	\Fib{k}+\Fib{k+1}&\Fib{k+1}\\\Fib{k+1}&\Fib{k}
\end{bmatrix}=&\begin{bmatrix}
	1&1\\ 1&0\\
\end{bmatrix}^{k+1}\\
	\begin{bmatrix}
	\Fib{k+2}&\Fib{k+1}\\\Fib{k+1}&\Fib{k}
\end{bmatrix}=&\begin{bmatrix}
	1&1\\ 1&0\\
\end{bmatrix}^{k+1}=G^{k+1}\\
	\end{align*}
Cvd.
\end{proof}
Possiamo dimostrare nuovamente l'identità Cassini~\vref{thm:FibCassini}
\begin{thm}[Identità di 
Cassini]\index{Fibonacci!identità!Cassini}
	Se $\Fib{n}$ è la successione di Fibonacci allora avremo:
\[\Fib{n+1}\Fib{n-1}-\Fib{n}^2=(-1)^n\]
\end{thm}
\begin{proof}
\begin{align*}
\begin{vmatrix}
	\Fib{n+1}&\Fib{n}\\\Fib{n}&\Fib{n-1}
\end{vmatrix}=&\Fib{n+1}\Fib{n-1}-\Fib{n}^2\\
\intertext{ma}
\abs{Q^n}=&\abs{Q}^n\\
\intertext{quindi, dato che}
\begin{vmatrix}
	1&1\\ 1&0\\
\end{vmatrix}=&-1
\end{align*}
Abbiamo la tesi.
\end{proof}
\section{Successioni generalizzate}
\begin{defn}[Successioni generalizzate]\index{Successione!generalizzata!matrice}
Se $\Gib{n}$ è la successione di Fibonacci generalizzata allora avremo:
\begin{equation}
H^n=\begin{bmatrix}
	\Gib{n+1}&\Gib{n}\\\Gib{n}&\Gib{n-1}
\end{bmatrix}
\end{equation}
\end{defn}
\begin{thm}[Forma matriciale 
generalizzata]\label{thm:Formamatricialegeneralizzata}\index{Fibonacci!generalizzata!matrice}
	Se $\Gib{n}$ è la successione di Fibonacci generalizzata allora avremo:
	\begin{align*}
		H^n=&\begin{bmatrix}
			\Gib{2}&\Gib{1}\\\Gib{1}&\Gib{0}
		\end{bmatrix}\begin{bmatrix}
			1&1\\ 1&0\\
		\end{bmatrix}^{n-1}\\
	=&\begin{bmatrix}
		h+k&k\\k&h\\
	\end{bmatrix}\begin{bmatrix}
		1&1\\ 1&0\\
	\end{bmatrix}^{n-1}\\
	\end{align*}
\end{thm}
\begin{proof}
Proviamolo per induzione su $n$
\begin{align*}
\intertext{Per $n=1$ è vero infatti:}
\begin{bmatrix}
	\Gib{2}&\Gib{1}\\\Gib{1}&\Gib{0}
\end{bmatrix}=&\begin{bmatrix}
		h+k&k\\k&h\\
	\end{bmatrix}\begin{bmatrix}
		1&1\\ 1&0\\
	\end{bmatrix}^{0}\\
=&\begin{bmatrix}
	h+k&k\\k&h\\
\end{bmatrix}\begin{bmatrix}
	1&0\\ 0&1\\
\end{bmatrix}\\
\intertext{Supponiamolo vero per $n$, proviamolo per $n+1$}
\begin{bmatrix}
	\Gib{n+1}&\Gib{n}\\\Gib{n}&\Gib{n-1}
\end{bmatrix}=&\begin{bmatrix}
	h+k&k\\k&h\\
\end{bmatrix}\begin{bmatrix}
	1&1\\ 1&0\\
\end{bmatrix}^{n-1}\\
\intertext{Moltiplico ambo i lati}
\begin{bmatrix}
	\Gib{n+1}&\Gib{n}\\\Gib{n}&\Gib{n-1}
\end{bmatrix}\begin{bmatrix}
1&1\\ 1&0\\
\end{bmatrix}=&\begin{bmatrix}
	h+k&k\\k&h\\
\end{bmatrix}\begin{bmatrix}
	1&1\\ 1&0\\
\end{bmatrix}^{n-1}\begin{bmatrix}
1&1\\ 1&0\\
\end{bmatrix}\\
\begin{bmatrix}
	\Gib{n+1}+\Gib{n}&\Gib{n+1}\\\Gib{n}+\Gib{n-1}&\Gib{n}
\end{bmatrix}=&\begin{bmatrix}
	h+k&k\\k&h\\
\end{bmatrix}\begin{bmatrix}
	1&1\\ 1&0\\
\end{bmatrix}^{n}\\
\begin{bmatrix}
	\Gib{n+2}&\Gib{n+1}\\\Gib{n+1}&\Gib{n}
\end{bmatrix}=&\begin{bmatrix}
	h+k&k\\k&h\\
\end{bmatrix}\begin{bmatrix}
	1&1\\ 1&0\\
\end{bmatrix}^{n}\\
\end{align*}
cvd.
\end{proof}
\begin{cor}[Derivazione]
	Se $\Gib{n}$ è la successione di Fibonacci generalizzata allora avremo:
\begin{equation}
	H^n=\begin{bmatrix}
		h+k&k\\k&h\\
	\end{bmatrix}Q^{n-1}
\end{equation}
\end{cor}
Anche in questo caso è possibile dimostrare di nuovo 
il~\vref{thm:FibCassini} 
\begin{thm}[Identità di Cassini 
generalizzata]\label{thm:identitàCassinigeneralizzatamatrici}
	Se $\Gib{n}$ è la successione di Fibonacci generalizzata allora 
	\begin{equation}
		\Gib{n-1}\cdot\Gib{n+1}-\Gib{n}^2=(-1)^n(k^2-hk-h^2)
	\end{equation}
\end{thm}\index{Fibonacci!identità!Cassini generalizzata}
\begin{proof}
Dal~\vref{thm:Formamatricialegeneralizzata} abbiamo:
\begin{align*}
	\det H^n=\begin{vmatrix}
		\Gib{n+1}&\Gib{n}\\\Gib{n}&\Gib{n-1}
	\end{vmatrix}=&\begin{vmatrix}
		h+k&k\\k&h\\
	\end{vmatrix}\begin{vmatrix}
		1&1\\ 1&0\\
	\end{vmatrix}^{n-1}\\
\Gib{n-1}\cdot\Gib{n+1}-\Gib{n}^2=&(h^2+hk-k^2)(-1)^{n-1}\\
=&(k^2-h^2-kh)(-1)^{n}\\
\end{align*}
cvd.
\end{proof}
% !TeX root = appuntifibo.tex
% !BIB TS-program = biber
% !TeX encoding = UTF-8
% !TeX spellcheck = it_IT
% !TeX program = lualatex
\chapter{Successioni generalizzate}
\section{Definizioni}
\begin{defn}[Successioni di Fibonacci generalizzata]
	Diremo successione di Fibonacci generalizzata la successione definita
	\begin{equation}
		\begin{cases}
			\FG{0}=0\\
			\FG{1}=1\\
			\FG{n}=p\FG{n-1}+\FG{n-2}\quad n>1\\
		\end{cases}
	\end{equation} 
\end{defn}\cite{Yalciner2013}
\begin{defn}[Successioni di Lucas generalizzata]
Diremo successione di Lucas generalizzata la successione definita
\begin{equation}
	\begin{cases}
		\LG{0}=2\\
		\LG{1}=p\\
		\LG{n}=p\LG{n-1}+\LG{n-2}\quad n>1\\
	\end{cases}
\end{equation}
\end{defn}\cite{Yalciner2013} 
\begin{lem}[Proprietà]
	Se $\alpha$ e $\beta$ sono le soluzioni dell'equazione \begin{equation}
		x^2-px-1=0
	\end{equation} allora:
	\begin{align*}
		\alpha\beta=&{}-1\\
		\alpha+\beta=&{}p\\
		\alpha-\beta=&{}\sqrt{p^2+4}\\
		p-2\beta=&{}\sqrt{p^2+4}\\
		2\alpha-p =&{}\sqrt{p^2+4}\\
	\end{align*}
\end{lem}  
\begin{thm}[Formula di Binet generalizzata]
	Se $\alpha$ e $\beta$ sono le soluzioni dell'equazione \begin{equation}
		x^2-px-1=0
	\end{equation} allora la formula di Binet per la successione
 generalizzata 
	di Fibonacci è \begin{equation}
	\FG{n}=\dfrac{\alpha^n-\beta^n}{\alpha-\beta}
\end{equation}
mentre \begin{equation}
	\LG{n}=\alpha^n+\beta^n
\end{equation}
è la formula di Binet per la successione generalizzata di Lucas 
\end{thm}\index{Formula!Binet!Fibonacci 
generalizzata}\index{Formula!Binet!Lucas generalizzata}\cite{Yalciner2013} 
\begin{proof}
	\begin{align*}
		\intertext{Poniamo:}
		&\FG{n}={}x\alpha^n+y\beta^n\\
		&\left\{
	\begin{aligned}
			\alpha^0x+\beta^0y=0\\ \alpha^1y+\beta^1y=1
		\end{aligned}\right.
		\intertext{Otteniamo}
		&\left\{
	\begin{aligned}
			x=&\dfrac{1}{\alpha-\beta}\\ 
			{\hspace{1.5em}}\\y=&-\dfrac{1}{\alpha-\beta}
		\end{aligned}\right.\\
		%\intertext{Per il~\vref{lem:FibpropPhi}}
		%&\left\{
		%\begin{aligned}
			%	x=\dfrac{a}{\alpha-\beta}\\ y=-\dfrac{b}{\alpha-\beta}
			%\end{aligned}\right.\\ 
			&\FG{n}={}\dfrac{\alpha^n-\beta^n}{\alpha-\beta}
		\end{align*}
		quindi
		\begin{equation}
			\FG{n}=\dfrac{1}{\sqrt{P^2+4}}
			\left[\left(\dfrac{p+\sqrt{p^2+4}}{2}\right)^n-\left(\dfrac{p-\sqrt{p^2+4}}{2}\right)^n\right]
		\end{equation}
		cvd.
	\end{proof}
%%%%%%%%%%%%%%%%%%%%%%%%%%%%%%%%%%%%%%%%%%%%%%%
\section{Definizioni}
\begin{defn}[Successioni di Fibonacci generalizzata]
	Diremo successione di Fibonacci generalizzata la successione definita
	\begin{equation}
		\begin{cases}
			\Fg{0}=0\\
			\Fg{1}=1\\
			\Fg{n}=p\Fg{n-1}-q\Fg{n-2}\quad n>1\\
		\end{cases}
	\end{equation} 
\end{defn}\cite{Rabinowitz_1996}
\begin{defn}[Successioni di Lucas generalizzata]
	Diremo successione di Lucas generalizzata la successione definita
	\begin{equation}
		\begin{cases}
			\Lg{0}=2\\
			\Lg{1}=p\\
			\Lg{n}=p\Lg{n-1}-q\Lg{n-2}\quad n>1\\
		\end{cases}
	\end{equation}
\end{defn}\cite{Rabinowitz_1996}
\begin{lem}[Proprietà]
	Se $r_{1}$ e $r_{2}$ sono le soluzioni dell'equazione \begin{equation}
		x^2-px+q=0
	\end{equation} allora:
	\begin{align*}
		r_{1}r_{2}=&{}q\\
		r_{1}+r_{2}=&{}p\\
		r_{1}-r_{2}=&{}\sqrt{p^2-4q}\\
		p-2r_{2}=&{}\sqrt{p^2-4q}\\
		2r_{1}-p =&{}\sqrt{p^2-4q}\\
	\end{align*}
\end{lem}  
\begin{thm}[Formula di Binet 
generalizzata]\label{thm:FormulaBinetgeneralizzatarr}
	Se $r_{1}$ e $r_{2}$ sono le soluzioni dell'equazione \begin{equation}
		x^2-px+q=0
	\end{equation} allora la formula di Binet per la successione
	generalizzata 
	di Fibonacci è \begin{equation}
		\Fg{n}=\dfrac{r_{1}^n-r_{2}^n}{r_{1}-r_{2}}
	\end{equation}
	mentre \begin{equation}
		\Lg{n}=r_{1}^n+r_{2}^n
	\end{equation}
	è la formula di Binet per la successione generalizzata di Lucas 
\end{thm}\index{Formula!Binet!Fibonacci 
	generalizzata}\index{Formula!Binet!Lucas 
	generalizzata}\cite{Rabinowitz_1996}
\begin{proof}
		\proofpart{Prima dimostrazione}
	\begin{align*}
		\intertext{Poniamo:}
		\Fg{n}=&{}xr_{1}^n+yr_{2}^n\\
		&\left\{
	\begin{aligned}
			r_{1}^0x+r_{2}^0y=0\\ r_{1}^1y+r_{2}^1y=1
		\end{aligned}\right.
		\intertext{Otteniamo}
		&\left\{
	\begin{aligned}
			x=&\dfrac{1}{r_{1}-r_{2}}\\ 
			y=&-\dfrac{1}{r_{1}-r_{2}}
		\end{aligned}\right.\\
		%\intertext{Per il~\vref{lem:FibpropPhi}}
		%&\left\{
		%\begin{aligned}
			%	x=\dfrac{a}{\alpha-\beta}\\ y=-\dfrac{b}{\alpha-\beta}
			%\end{aligned}\right.\\ 
			\Fg{n}=&{}\dfrac{r_{1}^n-r_{2}^n}{r_{1}-r_{2}}
		\end{align*}
		quindi
		\begin{equation}
			\Fg{n}=\dfrac{1}{\sqrt{p^2-4q}}
			\left[\left(\dfrac{p+\sqrt{p^2-4q}}{2}\right)^n
			-\left(\dfrac{p-\sqrt{p^2-4q}}{2}\right)^n\right]
		\end{equation}
			\proofpart{Seconda dimostrazione}
				\begin{align*}
				\intertext{Poniamo che:}
				\Lg{n}=&{}xr_{1}^n+yr_{2}^n\\
				&\left\{
				\begin{aligned}
					r_{1}^0x+r_{2}^0y=2\\ 
					r_{1}^1y+r_{2}^1y=p
				\end{aligned}\right.
				\intertext{Otteniamo}
				&\left\{
				\begin{aligned}
					x=&\dfrac{p-2r_{2}}{r_{1}-r_{2}}=1\\ 
					y=&-\dfrac{2r_{1}-p}{r_{1}-r_{2}}=1
				\end{aligned}\right.\\
				%\intertext{Per il~\vref{lem:FibpropPhi}}
				%&\left\{
				%\begin{aligned}
				%	x=\dfrac{a}{\alpha-\beta}\\ y=-\dfrac{b}{\alpha-\beta}
				%\end{aligned}\right.\\ 
				\Lg{n}={}&r_{1}^n+r_{2}^n
			\end{align*}
		quindi
		\begin{equation}
			\Lg{n}=
		\left(\dfrac{p+\sqrt{p^2-4q}}{2}\right)^n
			+\left(\dfrac{p-\sqrt{p^2-4q}}{2}\right)^n
		\end{equation}
	\end{proof}
\begin{thm}[Algoritmo per rimuovere $r_1$ e 
$r_2$]~\cite{Rabinowitz_1996}\label{thm:FibLucRimuovirr}
	Se $\Fg{n}$ è la successione di Fibonacci e  $\Lg{n}$ è quella di Lucas 
	allora:\begin{equation}
		\left\{\begin{aligned}
			r_1^n={}&\dfrac{\Lg{n}+(\sqrt{p^2-4q})\Fg{n}}{2}\\
			r_2^n={}&\dfrac{\Lg{n}-(\sqrt{p^2-4q})\Fg{n}}{2}\\
		\end{aligned}
		\right.
	\end{equation}
\end{thm}
\begin{proof}
	Utilizzando il~\vref{thm:FormulaBinetgeneralizzatarr} 
	possiamo scrivere
	\begin{equation*}
		\left\{
		\begin{aligned}
			\Fg{n}=&\dfrac{r_{1}^n-r_{2}^n}{r_{1}-r_{2}}\\
			\Lg{n}=&r_{1}^n+r_{2}^n
		\end{aligned}
		\right.
	\end{equation*}
	che risolto rispetto $r_1^n$ e $r_2^n$ da la tesi.
\end{proof}
% !TeX root = appuntifibo.tex
% !BIB TS-program = biber
% !TeX encoding = UTF-8
% !TeX spellcheck = it_IT
% !TeX program = lualatex
\chapter{Fibonacci}
\citaoeis{A000045}
\begin{longtable}{*{3}{l}}\toprule
\caption{Numero divisori}\\
\midrule
\endfirsthead
\multicolumn{3}{c} {\tablename\ \thetable\ -- \textit{Continua dalla pagina precedente}} \\
\toprule
\endhead
\bottomrule
\multicolumn{3}{r} {\textit{Continua nella pagina successiva}} \\
\endfoot
\endlastfoot
0&1&1\\
2&3&5\\
8&13&21\\
34&55&89\\
144&233&377\\
610&987&1597\\
2584&4181&6765\\
10946&17711&28657\\
46368&75025&121393\\
196418&317811&514229\\
832040&1346269&2178309\\
3524578&5702887&9227465\\
14930352&24157817&39088169\\
63245986&102334155&165580141\\
267914296&433494437&701408733\\
1134903170&1836311903&2971215073\\
4807526976&7778742049&12586269025\\
20365011074&32951280099&53316291173\\
86267571272&139583862445&225851433717\\
365435296162&591286729879&956722026041\\
1548008755920&2504730781961&4052739537881\\
6557470319842&10610209857723&17167680177565\\
27777890035288&44945570212853&72723460248141\\
117669030460994&190392490709135&308061521170129\\
498454011879264&806515533049393&1304969544928657\\
2111485077978050&3416454622906707&5527939700884757\\
8944394323791464&14472334024676221&23416728348467685\\
37889062373143906&61305790721611591&99194853094755497\\
160500643816367088&259695496911122585&420196140727489673\\
679891637638612258&1100087778366101931&1779979416004714189\\
2880067194370816120&4660046610375530309&7540113804746346429\\
12200160415121876738&1293530146158671551&13493690561280548289\\
14787220707439219840&9834167195010216513&6174643828739884737\\
16008811023750101250&3736710778780434371&1298777728820984005\\
5035488507601418376&6334266236422402381&11369754744023820757\\
17704020980446223138&10627031650760492279&9884308557497163801\\
2064596134548104464&11948904692045268265&14013500826593372729\\
7515661444929089378&3082418197812910491&10598079642741999869\\
13680497840554910360&5831833409587358613&1065587176432717357\\
6897420586020075970&7963007762452793327&14860428348472869297\\
4376692037216111008&790376311979428689&5167068349195539697\\
5957444661174968386&11124513010370508083&17081957671545476469\\
9759726608206432936&8394940206042357789&18154666814248790725\\
8102862946581596898&7810785687120836007&15913648633702432905\\
5277690247113717296&2744594807106598585&8022285054220315881\\
10766879861326914466&342420841837678731&11109300703164593197\\
11451721545002271928&4114278174457313509&15565999719459585437\\
1233533820207347330&16799533539666932767&18033067359874280097\\
16385856825831661248&15972180111996389729&13911292864118499361\\
11436728902405337474&6901277692814285219&18338006595219622693\\
6792540214324356296&6683802735834427373&13476342950158783669\\
1713401612283659426&15189744562442443095&16903146174726102521\\
13646146663458994000&12102548764475544905&7301951354224987289\\
957756044990980578&8259707399215967867&9217463444206948445\\
17477170843422916312&8247890213920313141&7278316983633677837\\
15526207197553990978&4357780107478117199&1437243231322556561\\
5795023338800673760&7232266570123230321&13027289908923904081\\
1812812405337582786&14840102314261486867&16652914719599069653\\
13046272960151004904&11252443606040522941&5851972492481976229\\
17104416098522499170&4509644517294923783&3167316542107871337\\
7676961059402795120&10844277601510666457&74494587203909961\\
10918772188714576418&10993266775918486379&3465294890923511181\\
14458561666841997560&17923856557765508741&13935674150897954685\\
13412786634953911810&8901716712142314879&3867759273386675073\\
12769475985528989952&16637235258915665025&10959967170735103361\\
9150458355941216770&1663681452966768515&10814139808907985285\\
12477821261874753800&4845216997073187469&17323038258947941269\\
3721511182311577122&2597805367549966775&6319316549861543897\\
8917121917411510672&15236438467273054569&5706816310975013625\\
2496510704538516578&8203327015513530203&10699837720052046781\\
456420661856025368&11156258381908072149&11612679043764097517\\
4322193351962618050&15934872395726715567&1810321673979782001\\
17745194069706497568&1108771669976727953&407221665973673905\\
1515993335950401858&1923215001924075763&3439208337874477621\\
5362423339798553384&8801631677673031005&14164055017471584389\\
4518942621435063778&236253565197096551&4755196186632160329\\
4991449751829256880&9746645938461417209&14738095690290674089\\
6037997555042539682&2329349171623662155&8367346726666201837\\
10696695898289863992&617298551246514213&11313994449536378205\\
11931293000782892418&4798543376609719007&16729836377392611425\\
3081635680292778816&1364727983975838625&4446363664268617441\\
5811091648244456066&10257455312513073507&16068546960757529573\\
7879258199561051464&5501061086609029421&13380319286170080885\\
434636299069558690&13814955585239639575&14249591884309198265\\
9617803395839286224&5420651206438932873&15038454602278219097\\
2012361735007600354&17050816337285819451&616433998583868189\\
17667250335869687640&18283684334453555829&17504190596613691853\\
17341130857357696066&16398577380261836303&15292964163909980753\\
13244797470462265440&10091017560662694577&4889070957415408401\\
14980088518078102978&1422415401783959763&16402503919862062741\\
17824919321646022504&15780679167798533629&15158854415735004517\\
12492789509823986530&9204899851849439431&3250945287963874345\\
12455845139813313776&15706790427777188121&9715891493880950281\\
6975937847948586786&16691829341829537067&5221023116068572237\\
3466108384188557688&8687131500257129925&12153239884445687613\\
2393627310993265922&14546867195438953535&16940494506432219457\\
13040617628161621376&11534368060884289217&6128241615336358977\\
17662609676220648194&5344107217847455555&4559972820358552133\\
9904080038206007688&14464052858564559821&5921388823061015893\\
1938697607916024098&7860086430977039991&9798784038893064089\\
17658870469870104080&9010910435053616553&8223036831214169017\\
17233947266267785570&7010240023772402971&5797443216330636925\\
12807683240103039896&158382382724125205&12966065622827165101\\
13124448005551290306&7643769554668903791&2321473486510642481\\
9965243041179546272&12286716527690188753&3805215495160183409\\
16091932022850372162&1450403444301003955&17542335467151376117\\
545994837742828456&18088330304894204573&187581068927481413\\
18275911373821685986&16748369039615783&18292659742861301769\\
18309408111900917552&18155323781052667705&18017987819244033641\\
17726567526587149730&17297811272121631755&16577634724999229869\\
15428701923411310008&13559592574700988261&10541550424402746653\\
5654398925394183298&16195949349796929951&3403604201481561633\\
1152809477568939968&4556413679050501601&5709223156619441569\\
10265636835669943170&15974859992289384739&7793752754249776293\\
5321868672829609416&13115621427079385709&18437490099908995125\\
13106367453278829218&13097113479478272727&7756736859047550329\\
2407106264816271440&10163843123863821769&12570949388680093209\\
4288048438834363362&16858997827514456571&2700302192639268317\\
1112555946444173272&3812858139083441589&4925414085527614861\\
8738272224611056450&13663686310138671311&3955214461040176145\\
17618900771178847456&3127371158509471985&2299527855978767825\\
5426899014488239810&7726426870467007635&13153325884955247445\\
2433008681712703464&15586334566667950909&18019343248380654373\\
15158933741339053666&14731532916010156423&11443722583639658473\\
7728511425940263280&725489935870370137&8454001361810633417\\
9179491297681003554&17633492659491636971&8366239883463088909\\
7552988469245174264&15919228352708263173&5025472748243885821\\
2497957027242597378&7523429775486483199&10021386802729080577\\
17544816578215563776&9119459307235092737&8217531811741104897\\
17336991118976197634&7107778857007750915&5998025902274396933\\
13105804759282147848&657086587846993165&13762891347129141013\\
14419977934976134178&9736125208395723575&5709359069662306137\\
15445484278058029712&2708099274010784233&18153583552068813945\\
2414938752370046562&2121778230729308891&4536716983099355453\\
6658495213828664344&11195212196928019797&17853707410756684141\\
10602175533975152322&10009138871022284847&2164570331287885553\\
12173709202310170400&14338279533598055953&8065244662198674737\\
3956780122087179074&12022024784285853811&15978804906373032885\\
9554085616949335080&7086146449612816349&16640232066562151429\\
5279634442465416162&3473122435318015975&8752756877783432137\\
12225879313101448112&2531892117175328633&14757771430276776745\\
17289663547452105378&13600690904019330507&12443610377761884269\\
7597557208071663160&1594423512123995813&9191980720195658973\\
10786404232319654786&1531640878805762143&12318045111125416929\\
13849685989931179072&7720987027347044385&3123928943568671841\\
10844915970915716226&13968844914484388067&6367016811690552677\\
1889117652465389128&8256134464155941805&10145252116621330933\\
18401386580777272738&10099894623689052055&10054537130756773177\\
1707687680736273616&11762224811493046793&13469912492229320409\\
6785393230012815586&1808561648532584379&8593954878545399965\\
10402516527077984344&549727331913832693&10952243858991817037\\
11501971190905649730&4007470976187915151&15509442167093564881\\
1070169069571928416&16579611236665493297&17649780306237421713\\
15782647469193363394&14985683701721233491&12321587097205045269\\
8860526725216727144&2735369748712220797&11595896473928947941\\
14331266222641168738&7480418622860565063&3364940771792182185\\
10845359394652747248&14210300166444929433&6608915487388125065\\
2372471580123502882&8981387067511627947&11353858647635130829\\
1888501641437207160&13242360289072337989&15130861930509545149\\
9926478145872331522&6610596002672325055&16537074148544656577\\
4700926077507430016&2791256152342534977&7492182229849964993\\
10283438382192499970&17775620612042464963&9612314920525413317\\
8941191458858326664&106762305674188365&9047953764532515029\\
9154716070206703394&18202669834739218423&8910641831236370201\\
8666567592266037008&17577209423502407209&7797032942058892601\\
6927498291851748194&14724531233910640795&3205285452052837373\\
17929816685963478168&2688358064306763925&2171430676560690477\\
4859788740867454402&7031219417428144879&11891008158295599281\\
475483502014192544&12366491660309791825&12841975162323984369\\
6761722748924224578&1156953837538657331&7918676586462881909\\
9075630424001539240&16994307010464421149&7623193360756408773\\
6170756297511278306&13793949658267687079&1517961882069413769\\
15311911540337100848&16829873422406514617&13695040889034063849\\
12078170237731026850&7326467053055539083&957893217077014317\\
8284360270132553400&9242253487209567717&17526613757342121117\\
8322123170842137218&7401992854474706719&15724116025316843937\\
4679364806081999040&1956736757689291361&6636101563771290401\\
8592838321460581762&15228939885231872163&5375034132982902309\\
2157229944505222856&7532264077488125165&9689494021993348021\\
17221758099481473186&8464508047765269591&7239522073537191161\\
15704030121302460752&4496808121130100297&1754094168723009433\\
6250902289853109730&8004996458576119163&14255898748429228893\\
3814151133295796440&18070049881725025333&3437456941311270157\\
3060762749326743874&6498219690638014031&9558982439964757905\\
16057202130602771936&7169440496857978225&4779898553751198545\\
11949339050609176770&16729237604360375315&10231832581260000469\\
8514326111910824168&299414619461273021&8813740731372097189\\
9113155350833370210&17926896082205467399&8593307359329285993\\
8073459367825201776&16666766727154487769&6293482021270137929\\
4513504674715074082&10806986695985212011&15320491370700286093\\
7680733992975946488&4554481289966680965&12235215282942627453\\
16789696572909308418&10578167782142384255&8921120281342141057\\
1052543989774973696&9973664271117114753&11026208260892088449\\
2553128458299651586&13579336719191740035&16132465177491391621\\
11265057822973580040&8950778926755420045&1769092676019448469\\
10719871602774868514&12488964278794316983&4762091807859633881\\
17251056086653950864&3566403820804033129&2370715833748432377\\
5937119654552465506&8307835488300897883&14244955142853363389\\
4106046557444709656&18351001700298073045&4010304184033231085\\
3914561810621752514&7924865994654983599&11839427805276736113\\
1317549726222168096&13156977531498904209&14474527257721072305\\
9184760715510424898&5212543899521945587&14397304615032370485\\
1163104440844764456&15560409055877134941&16723513496721899397\\
13837178478889482722&12113947901901830503&7504382307081761609\\
1171586135274040496&8675968442355802105&9847554577629842601\\
76778946276093090&9924333523905935691&10001112470182028781\\
1478701920378412856&11479814390560441637&12958516310938854493\\
5991586627789744514&503358865019047391&6494945492808791905\\
6998304357827839296&13493249850636631201&2044810134754918881\\
15538059985391550082&17582870120146468963&14674186031828467429\\
13810312078265384776&10037754036384300589&5401322040940133749\\
15439076077324434338&2393654044555016471&17832730121879450809\\
1779640092724915664&1165626140894814857&2945266233619730521\\
4110892374514545378&7056158608134275899&11167050982648821277\\
18223209590783097176&10943516499722366837&10719982016795912397\\
3216754442808727618&13936736459604640015&17153490902413367633\\
12643483288308456032&11350230117012272049&5546969331611176465\\
16897199448623448514&3997424706525073363&2447880081438970261\\
6445304787964043624&8893184869403013885&15338489657367057509\\
5784930453060519778&2676676036718025671&8461606489778545449\\
11138282526496571120&1153144942565564953&12291427469062136073\\
13444572411627701026&7289255806980285483&2287084144898434893\\
9576339951878720376&11863424096777155269&2993019974946324029\\
14856444071723479298&17849464046669803327&14259164044683731009\\
13661884017643982720&9474303988618162113&4689443932552593217\\
14163747921170755330&406447780013796931&14570195701184552261\\
14976643481198349192&11100095108673349837&7629994516162147413\\
283345551125945634&7913340067288093047&8196685618414038681\\
16110025685702131728&5859967230406618793&3523248842399198905\\
9383216072805817698&12906464915205016603&3842936914301282685\\
16749401829506299288&2145594670098030357&448252425894778029\\
2593847095992808386&3042099521887586415&5635946617880394801\\
8678046139767981216&14313992757648376017&4545294823706805617\\
412543507645630018&4957838331352435635&5370381838998065653\\
10328220170350501288&15698602009348566941&7580078105989516613\\
4831936041628531938&12412014147618048551&17243950189246580489\\
11209220263155077424&10006426378692106297&2768902568137632105\\
12775328946829738402&15544231514967370507&9872816388087557293\\
6970303829345376184&16843120217432933477&5366679973068758045\\
3763056116792139906&9129736089860897951&12892792206653037857\\
3575784222804384192&16468576429457422049&1597616578552254625\\
18066193008009676674&1217065512852379683&836514447152504741\\
2053579960004884424&2890094407157389165&4943674367162273589\\
7833768774319662754&12777443141481936343&2164467842092047481\\
14941910983573983824&17106378825666031305&13601545735530463513\\
12261180487486943202&7415982149307855099&1230418563085246685\\
8646400712393101784&9876819275478348469&76475914161898637\\
9953295189640247106&10029771103802145743&1536322219732841233\\
11566093323534986976&13102415543267828209&6221764793093263569\\
877436262651540162&7099201055744803731&7976637318396343893\\
15075838374141147624&4605731618827939901&1234825919259535909\\
5840557538087475810&7075383457347011719&12915940995434487529\\
1544580379071947632&14460521374506435161&16005101753578382793\\
12018879054375266338&9577236734244097515&3149371714909812237\\
12726608449153909752&15875980164063721989&10155844539508080125\\
7585080629862250498&17740925169370330623&6879261725523029505\\
6173442821183808512&13052704546706838017&779403294181094913\\
13832107840887932930&14611511135069027843&9996874902247409157\\
6161641963606885384&16158516865854294541&3873414755751628309\\
1585187547896371234&5458602303647999543&7043789851544370777\\
12502392155192370320&1099437933027189481&13601830088219559801\\
14701268021246749282&9856354035756757467&6110877983293955133\\
15967232019050712600&3631365928635116117&1151853873976277101\\
4783219802611393218&5935073676587670319&10718293479199063537\\
16653367155786733856&8924916561276245777&7131539643353428017\\
16056456204629673794&4741251774273550195&2350963905193672373\\
7092215679467222568&9443179584660894941&16535395264128117509\\
7531830775079460834&5620481965498026727&13152312740577487561\\
326050632365962672&13478363372943450233&13804414005309412905\\
8836033304543311522&4193703236143172811&13029736540686484333\\
17223439776829657144&11806432243806589861&10583127946926695389\\
3942816117023733634&14525944063950429023&22016107264611041\\
14547960171215040064&14569976278479651105&10671192375985139553\\
6794424580755239042&17465616956740378595&5813297463786066021\\
4832170346816893000&10645467810602959021&15477638157419852021\\
7676361894313259426&4707255978023559831&12383617872336819257\\
17090873850360379088&11027747648987646729&9671877425638474201\\
2252881000916569314&11924758426555043515&14177639427471612829\\
7655653780317104728&3386549134079165941&11042202914396270669\\
14428752048475436610&7024210889162155663&3006218863928040657\\
10030429753090196320&13036648617018236977&4620334296398881681\\
17656982913417118658&3830573136106448723&3040811975814015765\\
6871385111920464488&9912197087734480253&16783582199654944741\\
8249035213679873378&6585873339625266503&14834908553305139881\\
2974037819220854768&17808946372525994649&2336240118037297801\\
1698442416853740834&4034682534891038635&5733124951744779469\\
9767807486635818104&15500932438380597573&6821995851306864061\\
3876184215977910018&10698180067284774079&14574364283262684097\\
6825800276837906560&2953420486391039041&9779220763228945601\\
12732641249619984642&4065117939139378627&16797759188759363269\\
2416133054189190280&767148169239001933&3183281223428192213\\
3950429392667194146&7133710616095386359&11084140008762580505\\
18217850624857966864&10855246559910995753&10626353111059411001\\
3034855597260855138&13661208708320266139&16696064305581121277\\
11910528940191835800&10159849172063405461&3623634038545689645\\
13783483210609095106&17407117249154784751&12743856386054328241\\
11704229561499561376&6001341873844338001&17705571435343899377\\
5260169235478685762&4518996597113033523&9779165832591719285\\
14298162429704752808&5630584188586920477&1482002544582121669\\
7112586733169042146&8594589277751163815&15707176010920205961\\
5855021214961818160&3115453152172472505&8970474367134290665\\
12085927519306763170&2609657812731502219&14695585332038265389\\
17305243144769767608&13554084403098481381&12412583474158697373\\
7519923803547627138&1485763203996772895&9005687007544400033\\
10491450211541172928&1050393145376021345&11541843356917194273\\
12592236502293215618&5687335785500858275&18279572287794073893\\
5520163999585380552&5352992213669902829&10873156213255283381\\
16226148426925186210&8652560566470917975&6431964919686552569\\
15084525486157470544&3069746332134471497&18154271818291942041\\
2777274076716861922&2484801821299252347&5262075898016114269\\
7746877719315366616&13008953617331480885&2309087262937295885\\
15318040880268776770&17627128143206072655&14498424949765297809\\
13678809019261818848&9730489895317565041&4962554840869832273\\
14693044736187397314&1208855503347677971&15901900239535075285\\
17110755742882753256&14565911908708276925&13229923577881478565\\
9349091412880203874&4132270917052130823&13481362329932334697\\
17613633246984465520&12648251503207248601&11815140676482162505\\
6016648105979859490&17831788782462021995&5401692814732329869\\
4786737523484800248&10188430338217130117&14975167861701930365\\
6716854126209508866&3245277914201887615&9962132040411396481\\
13207409954613284096&4722797921315128961&17930207875928413057\\
4206261723533990402&3689725525752851843&7895987249286842245\\
11585712775039694088&1034955950616984717&12620668725656678805\\
13655624676273663522&7829549328220790711&3038429930784902617\\
10867979259005693328&13906409189790595945&6327644375086737657\\
1787309491167781986&8114953866254519643&9902263357422301629\\
18017217223676821272&9472736507389571285&9043209657356840941\\
69202091036860610&9112411748393701551&9181613839430562161\\
18294025587824263712&9028895353545274257&8876176867659986353\\
17905072221205260610&8334505015155695347&7792833162651404341\\
16127338177807099688&5473427266748952413&3154021370846500485\\
8627448637595452898&11781470008441953383&1962174572327854665\\
13743644580769808048&15705819153097662713&11002719660157919145\\
8261794739546030242&817770325994397771&9079565065540428013\\
9897335391534825784&530156383365702181&10427491774900527965\\
10957648158266230146&2938395859457206495&13896044017723436641\\
16834439877180643136&12283739821194528161&10671435624665619681\\
4508431372150596226&15179866996816215907&1241554295257260517\\
16421421292073476424&17662975587330736941&15637652805694661749\\
14853884319315847074&12044793051300957207&8451933296907252665\\
2049982274498658256&10501915571405910921&12551897845904569177\\
4607069343600928482&17158967189505497659&3319292459396874525\\
2031515575192820568&5350808034589695093&7382323609782515661\\
12733131644372210754&1668711180445174799&14401842824817385553\\
16070554005262560352&12025652756370394289&9649462687923403025\\
3228371370584245698&12877834058507648723&16106205429091894421\\
10537295413889991528&8196756769272334333&287308109452774245\\
8484064878725108578&8771372988177882823&17255437866902991401\\
7580066781371322608&6388760574564762393&13968827355936085001\\
1910843856791295778&15879671212727380779&17790515069518676557\\
15223442208536505720&14567213204345630661&11343911339172584765\\
7464380469808663810&361547735271696959&7825928205080360769\\
8187475940352057728&16013404145432418497&5754136012074924609\\
3320796083797791490&9074932095872716099&12395728179670507589\\
3023916201833672072&15419644381504179661&18443560583337851733\\
15416460891132479778&15413277400760779895&12382994218183708057\\
9349527545234936336&3285777689709092777&12635305234944029113\\
15921082924653121890&10109644085887599387&7583982936831169661\\
17693627022718769048&6830865885840387093&6077748834849604525\\
12908614720689991618&539619481830044527&13448234202520036145\\
13987853684350080672&8989343813160565201&4530453423801094257\\
13519797236961659458&18050250660762753715&13123303824014861557\\
12726810411068063656&7403370161373373597&1683436498731885637\\
9086806660105259234&10770243158837144871&1410305745232852489\\
12180548904069997360&13590854649302849849&7324659479663295593\\
2468770055256593826&9793429534919889419&12262199590176483245\\
3608885051386821048&15871084641563304293&1033225619240573725\\
16904310260803878018&17937535880044451743&16395102067138778145\\
15885893873473678272&13834251866902904801&11273401666667031457\\
6660909459860384642&17934311126527416099&6148476512678249125\\
5636043565496113608&11784520078174362733&17420563643670476341\\
10758339648135287458&9732159218096212183&2043754792521948025\\
11775914010618160208&13819668803140108233&7148838740048716825\\
2521763469479273442&9670602209527990267&12192365679007263709\\
3416223814825702360&15608589493832966069&578069234949116813\\
16186658728782082882&16764727963731199695&14504642618803730961\\
12822626508825379040&8880525053919558385&3256407489035385809\\
12136932542954944194&15393340031990330003&9083528501235722581\\
6030124459516500968&15113652960752223549&2697033346559172901\\
17810686307311396450&2060975580161017735&1424917813762862569\\
3485893393923880304&4910811207686742873&8396704601610623177\\
13307515809297366050&3257476337198437611&16564992146495803661\\
1375724409984689656&17940716556480493317&869696892755631357\\
363669375526573058&1233366268282204415&1597035643808777473\\
2830401912090981888&4427437555899759361&7257839467990741249\\
11685277023890500610&496372418171690243&12181649442062190853\\
12678021860233881096&6412927228586520333&644205015110849813\\
7057132243697370146&7701337258808219959&14758469502505590105\\
4013062687604258448&324788116400296937&4337850804004555385\\
4662638920404852322&9000489724409407707&13663128644814260029\\
4216874295514116120&17880002940328376149&3650133162132940653\\
3083392028751765186&6733525190884705839&9816917219636471025\\
16550442410521176864&7920615556448096273&6024313893259721521\\
13944929449707817794&1522499269257987699&15467428718965805493\\
16989927988223793192&14010612633480047069&12553796547994288645\\
8117665107764784098&2224717582049521127&10342382689814305225\\
12567100271863826352&4462738887968579961&17029839159832406313\\
3045833974091434658&1628929060214289355&4674763034305724013\\
6303692094520013368&10978455128825737381&17282147223345750749\\
9813858278461936514&8649261428098135647&16375632850520545\\
8665637060948656192&\\
\bottomrule\end{longtable}

			

\nocite{*}
 \addcontentsline{toc}{chapter}{\bibname}
\printbibliography
 \addcontentsline{toc}{chapter}{\indexname}
 \printindex
 \appendix
 \chapter{Mezzi usati}
 \CDMezziUsati
\end{document}
